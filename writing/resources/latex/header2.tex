\usepackage{tikz, subfig, amsthm, multirow, float, enumerate}
\usepackage{tikz-3dplot}
\usetikzlibrary{arrows, shapes, positioning}
\usepackage{bm}

%%INCLUDED-AK
%%DELETE HERE
\usepackage{amsthm}
\newtheoremstyle{theorem}{6pt}{6pt}{\rm}{}{\sffamily}{ }{ }{}
\theoremstyle{theorem}
\newtheorem{theorem}{\sc Theorem}[section]

\newtheoremstyle{lemma}{6pt}{6pt}{\rm}{}{\sffamily}{ }{ }{}
\theoremstyle{lemma}
\newtheorem{lemma}{\sc Lemma}[section]

\newtheorem{acknowledgement}[theorem]{Acknowledgement}
\newtheorem{algorithm}[theorem]{Algorithm}
\newtheorem{axiom}[theorem]{Axiom}
\newtheorem{case}[theorem]{Case}
\newtheorem{claim}[theorem]{Claim}
\newtheorem{conclusion}[theorem]{Conclusion}
\newtheorem{condition}[theorem]{Condition}
\newtheorem{conjecture}[theorem]{Conjecture}
\newtheorem{criterion}[theorem]{Criterion}

\newtheoremstyle{example}{6pt}{6pt}{\rm}{}{\sffamily}{ }{ }{}
\theoremstyle{example}
\newtheorem{example}[theorem]{\sc Example}

\newtheoremstyle{corollary}{6pt}{6pt}{\rm}{}{\sffamily}{ }{ }{}
\theoremstyle{corollary}
\newtheorem{corollary}{\sc Corollary}[section]

\newtheoremstyle{definition}{6pt}{6pt}{\rm}{}{\sffamily}{ }{ }{}
\theoremstyle{definition}
\newtheorem{definition}[theorem]{\sc Definition}

\newtheorem{exercise}[theorem]{Exercise}

\newtheorem{notation}[theorem]{Notation}
\newtheorem{problem}[theorem]{Problem}
\newtheorem{proposition}[theorem]{\sc Proposition}

\newtheoremstyle{remark}{6pt}{6pt}{\rm}{}{\sffamily}{ }{ }{}
\theoremstyle{remark}
\newtheorem{remark}{\sc Remark}[section]
\newtheorem{solution}[theorem]{Solution}
\newtheorem{summary}[theorem]{Summary}

\newtheoremstyle{approximation}{6pt}{6pt}{\rm}{}{\sffamily}{ }{ }{}
\theoremstyle{approximation}
\newtheorem{approximation}{\sc Approximation}

\newtheoremstyle{scheme}{6pt}{6pt}{\rm}{}{\sffamily}{ }{ }{}
\theoremstyle{scheme}
\newtheorem{scheme}{\sc Scheme}

% \newtheorem{proposition}{Proposition}
% \newtheorem{lemma}{Lemma}
% \newtheorem{corollary}{Corollary}

% \theoremstyle{definition}
% \newtheorem{definition}{Definition}
% \newtheorem{assumption}{Assumption}
% \newtheorem{remark}{Remark}
%% STOP DELETING

\DeclareMathOperator*{\argmin}{arg\,min}
\DeclareMathOperator*{\argmax}{arg\,max}

\newcommand{\REP}{\mathrm{LREP}}
\newcommand{\DN}{\Delta_N}

\newcommand{\ma}{\mathrm{max}_{\boldsymbol \theta_N}}
\newcommand{\mi}{\mathrm{min}_{\boldsymbol \theta_N}}


\newcommand{\maa}{\mathrm{max}_{-\boldsymbol \theta_N}}
\newcommand{\mii}{\mathrm{min}_{-\boldsymbol \theta_N}}


\newcommand{\nv}{{n_{\scriptscriptstyle V}}}
\newcommand{\nh}{{n_{\scriptscriptstyle H}}}
\newcommand{\E}{E}

\newcommand{\elt}{A_{N}(\boldsymbol \theta_N) }
\newcommand{\Gam}{B_{N}(\boldsymbol \theta_N) }

\newcommand{\Gamc}{C_{N}(\boldsymbol \theta_N) }
\newcommand{\Gamt}{\Gamma_{N,2}(\boldsymbol \theta_N) }

