\documentclass[]{article}
%\usepackage{nips15submit_e,times}
\usepackage{url}

\providecommand{\tightlist}{%
  \setlength{\itemsep}{0pt}\setlength{\parskip}{0pt}}

\usepackage{lmodern}
\usepackage{amssymb,amsmath}
\usepackage{ifxetex,ifluatex}
\usepackage{fixltx2e} % provides \textsubscript
\ifnum 0\ifxetex 1\fi\ifluatex 1\fi=0 % if pdftex
  \usepackage[T1]{fontenc}
  \usepackage[utf8]{inputenc}
\else % if luatex or xelatex
  \ifxetex
    \usepackage{mathspec}
    \usepackage{xltxtra,xunicode}
  \else
    \usepackage{fontspec}
  \fi
  \defaultfontfeatures{Mapping=tex-text,Scale=MatchLowercase}
  \newcommand{\euro}{€}
\fi
% use upquote if available, for straight quotes in verbatim environments
\IfFileExists{upquote.sty}{\usepackage{upquote}}{}
% use microtype if available
\IfFileExists{microtype.sty}{%
\usepackage{microtype}
\UseMicrotypeSet[protrusion]{basicmath} % disable protrusion for tt fonts
}{}
\usepackage{longtable,booktabs}
\usepackage{graphicx}
\makeatletter
\def\maxwidth{\ifdim\Gin@nat@width>\linewidth\linewidth\else\Gin@nat@width\fi}
\def\maxheight{\ifdim\Gin@nat@height>\textheight\textheight\else\Gin@nat@height\fi}
\makeatother
% Scale images if necessary, so that they will not overflow the page
% margins by default, and it is still possible to overwrite the defaults
% using explicit options in \includegraphics[width, height, ...]{}
\setkeys{Gin}{width=\maxwidth,height=\maxheight,keepaspectratio}
\ifxetex
  \usepackage[setpagesize=false, % page size defined by xetex
              unicode=false, % unicode breaks when used with xetex
              xetex]{hyperref}
\else
  \usepackage[unicode=true]{hyperref}
\fi
\hypersetup{breaklinks=true,
            bookmarks=true,
            pdfauthor={},
            pdftitle={A note on the instability and degeneracy of deep learning models},
            colorlinks=true,
            citecolor=blue,
            urlcolor=blue,
            linkcolor=magenta,
            pdfborder={0 0 0}}
\urlstyle{same}  % don't use monospace font for urls
\setlength{\parindent}{0pt}
\setlength{\parskip}{6pt plus 2pt minus 1pt}
\setlength{\emergencystretch}{3em}  % prevent overfull lines
\setcounter{secnumdepth}{5}

% Tech stuff starts here
\usepackage{psfrag,epsf}
%\pdfminorversion=4

\usepackage{tikz, subfig, amsthm, multirow, float, enumerate}
\usepackage{tikz-3dplot}
\usetikzlibrary{arrows, shapes, positioning}
\usepackage{bm}

%%INCLUDED-AK
\usepackage{amsthm}
\newtheorem{theorem}{Theorem}
\newtheorem{proposition}{Proposition}
\newtheorem{lemma}{Lemma}
\newtheorem{corollary}{Corollary}

\theoremstyle{definition}
\newtheorem{definition}{Definition}
\newtheorem{assumption}{Assumption}
\newtheorem{remark}{Remark}

\DeclareMathOperator*{\argmin}{arg\,min}
\DeclareMathOperator*{\argmax}{arg\,max}

\newcommand{\REP}{\mathrm{LREP}}
\newcommand{\DN}{\Delta_N}

\newcommand{\ma}{\mathrm{max}_{\boldsymbol \theta_N}}
\newcommand{\mi}{\mathrm{min}_{\boldsymbol \theta_N}}


\newcommand{\maa}{\mathrm{max}_{-\boldsymbol \theta_N}}
\newcommand{\mii}{\mathrm{min}_{-\boldsymbol \theta_N}}


\newcommand{\nv}{{n_{\scriptscriptstyle V}}}
\newcommand{\nh}{{n_{\scriptscriptstyle H}}}
\newcommand{\E}{E}

\newcommand{\elt}{A_{N}(\boldsymbol \theta_N) }
\newcommand{\Gam}{B_{N}(\boldsymbol \theta_N) }

\newcommand{\Gamc}{C_{N}(\boldsymbol \theta_N) }
\newcommand{\Gamt}{\Gamma_{N,2}(\boldsymbol \theta_N) }
\bigskip


% DON'T change margins - should be 1 inch all around.
\addtolength{\oddsidemargin}{-.5in}%
\addtolength{\evensidemargin}{-.5in}%
\addtolength{\textwidth}{1in}%
\addtolength{\textheight}{1.3in}%
\addtolength{\topmargin}{-.8in}%

\let\BeginKnitrBlock\begin \let\EndKnitrBlock\end
\begin{document}

\def\spacingset#1{\renewcommand{\baselinestretch}%
{#1}\small\normalsize} \spacingset{1}

\bigskip
\bigskip
\bigskip
%\begin{center}
  %{\LARGE\bf Title}
  \title{\bf A note on the instability and degeneracy of deep learning models}
%\end{center}
\medskip

\maketitle
\begin{abstract}
A probability model exhibits instability if small changes in a data
outcome result in large, and often unanticipated, changes in
probability. This instability is a property of the probability model,
rather than the fitted parameter vector. For correlated data structures
found in several application areas, there is increasing interest in
predicting/identifying such sensitivity in model probability structure.
We consider the problem of quantifying instability for general
probability models defined on sequences of observations, where each
sequence of length \(N\) has a finite number of possible values. A
sequence of probability models results, indexed by \(N\), that
accommodates data of expanding dimension. Model instability is formally
shown to occur when a certain log-probability ratio under such models
grows faster than \(N\). In this case, a one component change in the
data sequence can shift probability by orders of magnitude. Also, as
instability becomes more extreme, the resulting probability models are
shown to tend to degeneracy, placing all their probability on
potentially small portions of the sample space. These results on
instability apply to large classes of models commonly used in random
graphs, network analysis, and machine learning contexts.
\end{abstract}
\noindent%
{\it Keywords:}  Degeneracy, Instability, Classification, Deep Learning, Graphical Models
\vfill

\hfill {\tiny technometrics tex template (do not remove)}
\newpage
\spacingset{1.45} % DON'T change the spacing!


\section{Introduction}\label{introduction}

We consider the behavior, and the potential impropriety, of sequences of
discrete probability models built to incorporate observations of
increasing sample size \(N\). Interest is in identifying instability in
such models, which is roughly characterized by probabilities with
extreme sensitivity to small changes in data configuration. The concept
of instability was introduced in the field of statistical physics (i.e.,
point processes) by Ruelle
(\protect\hyperlink{ref-ruelle1999statistical}{1999}) and then further
extended by Schweinberger
(\protect\hyperlink{ref-schweinberger2011instability}{2011}) for a
family of exponential models. At issue, models exhibiting instability
are typically undesirable as these tend to provide poor representations
of data or data-generation. As an example, such models can include
near-degenerate distributions that assign all essential probability mass
to only a subset of an overall sample space. The latter issue in
connection to degeneracy has been recognized as a concern in that
dominant model outcomes may not resemble observed data (cf. Handcock
\protect\hyperlink{ref-handcock2003assessing}{2003}). As a compounding
issue, model instability often has direct negative impacts for
statistical inference and computations based on likelihood functions.
Namely, volatilities in probability structure can potentially hamper the
numerical evaluations required for maximum likelihood estimation as well
as other model-based simulations via Markov Chain Monte Carlo (MCMC).
These reasons motivate our general study of instability for a broad
class of probability models, described next.

In the model framework, let \(\boldsymbol X_N = (X_1, \dots, X_N)\)
denote a collection of discrete random variables with a finite sample
space, \(\mathcal{X}^N\), represented as some \(N\)-fold Cartesian
product. That is, \(\mathcal{X}\) with \(|\mathcal{X}| < \infty\)
denotes the set of potential outcomes for each single variable \(X_i\),
so that the product space \(\mathcal{X}^N\) corresponds to values for
the variables \(\boldsymbol X_N=(X_1,\ldots,X_N)\). For each \(N\), let
\(P_{\boldsymbol \theta_N}\) denote a probability model on
\(\mathcal{X}^N\), under which
\(P_{\boldsymbol \theta_N}(x_1, \dots, x_N) > 0\) is the probability of
the data outcome \((x_1, \dots, x_N) \in \mathcal{X}^N\). In this, we
assume that the model support of \(P_{\boldsymbol \theta_N}\) is the
sample space \(\mathcal{X}^N\). This framework produces probability
models \(P_{\boldsymbol \theta_N}\), indexed by a generic sequence of
parameters \(\boldsymbol \theta_N\), to describe data
\(\boldsymbol X_N\) of any given sample size \(N \geq 1\). For
simplicity, we will refer to this distributional class as \emph{Finite
Outcome Everywhere Supported (FOES)} models in the following. The
dimension and structure of such parameters are generic, without
restriction, though natural cases will be seen to include those where
\(\boldsymbol \theta_N \in \mathbb{R}^{q(N)}\) for some arbitrary
integer-valued function \(q(\cdot) \geq 1\).

Section \ref{examples} provides some examples of FOES models encountered
in graph/network analysis and machine learning (i.e., deep learning
models). These are used as references for later illustrations. Section
\ref{instability-results} then establishes several formal results for
FOES models with regard to instability. Schweinberger
(\protect\hyperlink{ref-schweinberger2011instability}{2011}) originally
developed instability results specific to a certain class of discrete
exponential models. For similar exponential models with random networks,
Handcock (\protect\hyperlink{ref-handcock2003assessing}{2003}) studied
model degeneracy, where a probability model places near complete mass on
modes and may thereby narrow the feasible model outcomes. As findings
here and from Schweinberger
(\protect\hyperlink{ref-schweinberger2011instability}{2011}) suggest,
model instability and degeneracy may also be related by viewing
degeneracy as an extreme, or limiting form, of instability. Our main
results establish a broad characterization of model instability,
appropriate across the whole FOES model class, that incorporates results
of Schweinberger
(\protect\hyperlink{ref-schweinberger2011instability}{2011}) as a
special case. We prescribe a general and simple condition for
identifying instability in a FOES model sequence, which quantifies
whether certain maximal probabilities in a FOES model are too extreme
relative to the sample size \(N\). When fulfilled, the probability
structure of a FOES model is shown to exhibit extreme sensitivity, with
probability assignments possessing extreme peaks and troughs across
nearly identical outcomes. As the measure of model instability
increases, probabilities from an unstable FOES model additionally
increase in volatility and provably slide into degeneracy. Section
\ref{implications} then emphasizes the implications of such model
instability, showing that such impropriety can be expected to
numerically hinder maximum likelihood estimation and MCMC-based
simulations. As one potential remedy, suggestions are given or
constraining model parameterizations to avoid the most problematic
regions of the parameter space. Proofs of the main results appear in
Appendix \ref{appendix-instab}.

\section{Examples}\label{examples}

Many model families fall under the umbrella of FOES models. For
illustration, this section presents three specific examples of FOES
models, including models with deep architectures.

\subsection{Discrete Exponential Family
Models}\label{discrete-exponential-family-models}

For random variables
\(\boldsymbol X \equiv\boldsymbol X_N= (X_1, \dots, X_N)\) with sample
space \(\mathcal{X}^N\), \(|\mathcal{X}| < \infty\), consider an
exponential family model for \(\boldsymbol X\) with probability mass
function given by
\begin{equation}
\label{eq:expo}
p_{N, \boldsymbol \theta}(\boldsymbol x) = \exp\left[\boldsymbol\eta^T(\boldsymbol \theta) \boldsymbol g_N(\boldsymbol x) - \psi(\boldsymbol \theta)\right], \qquad \boldsymbol x \in \mathcal{X}^N,
\end{equation}
depending on parameter vector
\(\boldsymbol \theta \in \Theta_N \subset \mathbb{R}^{k}\) and natural
parameter function
\(\boldsymbol \eta : \mathbb{R}^k \mapsto \mathbb{R}^L\) with fixed
positive integers \(k\) and \(L\) denoting their dimensions. Above,
\(\boldsymbol g_N : \mathcal{X}^N \mapsto \mathbb{R}^L\) is a vector of
sufficient statistics, while \[
\psi(\boldsymbol \theta) = \log \sum\limits_{\boldsymbol x \in \mathcal{X}^N}\exp\left[\boldsymbol \eta^T(\boldsymbol \theta) \boldsymbol g_N(\boldsymbol x) \right], \qquad \boldsymbol \theta \in \Theta_N\equiv \{\boldsymbol \theta \in \mathbb{R}^k : \psi(\boldsymbol \theta) < \infty \},
\] denotes the normalizing function with parameter space \(\Theta_N\).
The natural parameter function \(\eta (\boldsymbol \theta)\) has a
linear form (i.e.,
\(\eta (\boldsymbol \theta)= \bm{A} \boldsymbol \theta\) for a given
\(L \times k\) matrix \(\bm{A}\)) in many common model formulations,
though may also be nonlinear (e.g., curved exponential families). In the
linear case, \(\eta (\boldsymbol \theta) = \boldsymbol \theta\) may be
generally assumed in the exponential parameterization with a minor
modification to the definition of sufficient statistics
\(\boldsymbol g_N(\boldsymbol x)\).

Such discrete exponential family models are special cases of the FOES
models, as seen by defining
\(P_{\boldsymbol \theta_N}(\boldsymbol x)\equiv p_{N,\boldsymbol \theta_N}(\boldsymbol x)> 0\),
\(\boldsymbol x \in \mathcal{X}^N\), based on \eqref{eq:expo} and a
parameter sequence
\(\boldsymbol \theta_N \in \Theta_N \subset \mathbb{R}^k\). For example,
if observations \(\boldsymbol X = (X_1,\ldots,X_N)\) correspond to \(N\)
independent and identically distributed Bernoulli random variables, each
indicating a binary \(0\)-\(1\) outcome, the resulting probabilities
have exponential form \eqref{eq:expo} given by
\begin{equation}
\label{eq:mod1}
P_{\boldsymbol \theta_N}(\boldsymbol x) \propto
 \exp\left[\boldsymbol \theta_N \sum_{i=1}^N x_i\right], \qquad \boldsymbol x=(x_1,\ldots,x_N) \in\{0,1\}^N, 
 \end{equation}
with sufficient statistic
\(\boldsymbol g_N(\boldsymbol x)\equiv \sum_{i=1}^N x_i\) and ``log odds
ratio'' parameter
\(\boldsymbol \theta_N \equiv \log[ P_{\boldsymbol \theta_N}(X_i=1)/P_{\boldsymbol \theta_N}(X_i=0) ] \in \mathbb{R}\).
More generally, supposing \(\boldsymbol X =(X_1,\ldots,X_N)\) represent
\(N\) independent trials, each assuming an outcome \(\{1,\ldots,k\}\)
among \(k\) possibilities (e.g., a die roll), a multinomial distribution
is given by
\begin{equation}
\label{eq:mod11}
P_{\boldsymbol \theta_N}(\boldsymbol x) \propto  \exp\left[  \boldsymbol \theta_{N}^T g_N(\boldsymbol x)   \right] =
\exp\left[ \sum_{j=1}^k {\theta_{j,N}} \sum_{i=1}^N \mathbb{I}(x_i=j) \right], \qquad \boldsymbol x  \in\{1,\ldots,k\}^N, 
\end{equation}
with sufficient statistic \(\boldsymbol g_N(\boldsymbol x)\) involving a
count \(\sum_{i=1}^N \mathbb{I}(x_i=j)\) for each outcome
\(j \in \{1,\ldots,k\}\), where \(\mathbb{I}(\cdot)\) denotes the
indicator function, and parameters
\(\boldsymbol \theta_N=(\theta_{1,N},\ldots,\theta_{k,N})\in\mathbb{R}^k\)
defining log-probability ratios
\(\theta_{i,N}-\theta_{j,N} =\log [P_{\boldsymbol \theta_N}(X_1=i)/P_{\boldsymbol \theta_N}(X_1=j)]\).
In addition to such standard models for discrete independent data,
exponential models of FOES type commonly arise with dependent spatial
data (Besag \protect\hyperlink{ref-besag1974spatial}{1974}) and
network/relational data (Wasserman and Faust
\protect\hyperlink{ref-wasserman1994social}{1994}; Handcock
\protect\hyperlink{ref-handcock2003assessing}{2003}). For a random graph
or network with, say, \(n\) nodes, consider \(N={n \choose 2}\) random
edges where the \(i\)th edge is associated with a pair of nodes
\(s_i \equiv \{v_i,u_i\}\) and a binary variable \(X_i\in\{0,1\}\)
indicating presence/absence of an edge among the node pair \(s_i\),
\(i=1,\ldots,N\). Here the length \(N\) of the edge variable sequence
\(\boldsymbol X = (X_1,\ldots,X_N)\) increases as a function of node
number \(n\) and corresponding exponential models often incorporate
graph topographical features derived from \(\boldsymbol X\). As an
example, consider a graph model of exponential/FOES form prescribed by
\begin{equation}
\label{eq:mod2}
P_{\boldsymbol \theta_N}(\boldsymbol x) \propto
 \exp\left[\sum_{j=1}^3 \theta_{j,N} g_{j,N}(\boldsymbol x)\right], \quad\qquad \boldsymbol x=(x_1,\ldots,x_N)  \in\{0,1\}^N,
\end{equation}
\[
  g_{1,N}(\boldsymbol x) \equiv \sum_{i=1}^N  x_i, \qquad\quad g_{2,N}(\boldsymbol x) \equiv \sum_{1 \leq i<j \leq N,\atop s_i \cap s_j \neq \emptyset}\!\!\!   x_i x_j, \qquad
  g_{3,N}(\boldsymbol x) \equiv \sum_{1 \leq i<j<\ell \leq N, \atop s_i \cap s_j \neq \emptyset,s_i \cap s_\ell \neq \emptyset, \\ s_j \cap s_\ell \neq \emptyset } \!\!\!\!\!\!\!\!  \!\!\!\!\!\!\!\!   x_i x_j x_\ell,
\] involving the numbers of edges, 2-stars and triangles among an
outcome \(\boldsymbol x\) given by \(g_{1,N}(\boldsymbol x)\),
\(g_{2,N}(\boldsymbol x)\) and \(g_{3,N}(\boldsymbol x)\), respectively,
along with \(k=3\) real parameters
\(\boldsymbol \theta_N \equiv (\theta_{1,N},\theta_{2,N},\theta_{3,N})\).
For this network model \eqref{eq:mod2} in particular, as well as for more
general models of form \eqref{eq:expo}, Schweinberger
(\protect\hyperlink{ref-schweinberger2011instability}{2011}) considered
instability in such exponential models with sequences of fixed
parameters
\(\boldsymbol \theta_N = (\theta_1,\ldots,\theta_k)\in\mathbb{R}^k\),
\(N \geq 1\), of fixed dimension \(k\).

For model sequences
\(P_{\boldsymbol \theta_N}(\boldsymbol x)\equiv p_{N,\boldsymbol \theta_N}(\boldsymbol x)\)
of the exponential type \eqref{eq:expo}, such as those in
\eqref{eq:mod1}-\eqref{eq:mod2}, note that the dimension \(k\) of the
parameter \(\boldsymbol \theta_N\in\Theta \subset \mathbb{R}^k\)
necessarily remains the same for all sample sizes \(N \geq 1\) as the
form of the natural parameter function \(\eta(\cdot)\) in \eqref{eq:expo}
and the number of sufficient statistics
\(\boldsymbol g_{N}(\boldsymbol x)\) do not depend on \(N\).
Consequently, \(\boldsymbol \theta_N\) lies in a parameter space of
fixed Euclidean dimension \(k\). However, this aspect need not be true
for other types of FOES models considered in Sections \ref{rbm} -
\ref{deep-learning}, where instead the numbers of parameters and
sufficient statistics commonly increase with the sample size \(N\).

\subsection{Restricted Boltzmann Machines}\label{rbm}

A restricted Boltzmann machine (RBM) is an undirected graphical model
specified for discrete or continuous random variables, with binary
variables being most common (cf. Smolensky
\protect\hyperlink{ref-smolensky1986information}{1986}). A RBM
architecture has two layers, hidden (\(\mathcal{H}\)) and visible
(\(\mathcal{V}\)), with conditional independence within each layer. Let
\(\boldsymbol X = (X_1,\ldots,X_N)\) denote the \(N\) random variables
for visibles with support \(\mathcal{X}^N\) and
\(\boldsymbol H = (H_1,\ldots,H_{N_\mathcal{H}})\) denote the
\(N_\mathcal{H}\) random variables for hiddens with support
\(\mathcal{X}^{N_\mathcal{H}}\) where \(\mathcal{X} = \{-1,1\}\). For
parameters
\(\boldsymbol \theta_N^{\mathcal{H}} \in \mathbb{R}^{N_\mathcal{H}}\),
\(\boldsymbol \theta_N^{\mathcal{V}}\in \mathbb{R}^N\), and
\(\boldsymbol \theta_N^{\mathcal{HV}}\) as a real matrix with dimension
\(N_\mathcal{H} \times N\), the RBM model for
\(\tilde{\boldsymbol X}=(\boldsymbol X,\boldsymbol H)\) has the joint
probability mass function
\begin{equation}
\label{eq:RBM1}
\tilde{P}_{\boldsymbol \theta_N} (\tilde{\boldsymbol x}) = \exp\left[ (\boldsymbol \theta_N^{\mathcal{H}})^T \boldsymbol h + \boldsymbol (\boldsymbol \theta_N^{\mathcal{V}})^T \boldsymbol x + \boldsymbol h^T  \boldsymbol\theta_N^{\mathcal{HV}} \boldsymbol x - \psi(\boldsymbol \theta_N)\right], \quad \tilde{\boldsymbol x} = (\boldsymbol x, \boldsymbol h) \in \{\pm 1\}^{N+N_\mathcal{H}}
\end{equation}
with normalizing function \[
\psi(\boldsymbol \theta_N) = \log \sum_{\tilde{\boldsymbol x} \in \{\pm 1\}^{N+N_H} } \exp\left[ (\boldsymbol \theta_N^{\mathcal{H}})^T \boldsymbol h + \boldsymbol (\boldsymbol \theta_N^{\mathcal{V}})^T \boldsymbol x + \boldsymbol h^T  \boldsymbol\theta_N^{\mathcal{HV}} \boldsymbol x\right].
\] Let
\(\boldsymbol \theta_N = (\boldsymbol \theta_N^{\mathcal{H}}, \boldsymbol \theta_N^{\mathcal{V}}, \boldsymbol\theta_N^{\mathcal{HV}} ) \in \Theta_N \equiv \mathbb{R}^{q(N)}\),
with \(q(N) = N + N_\mathcal{H} + N*N_\mathcal{H}\), denote the
parameter vector for the RBM, as indexed by the number \(N\) of visible
random variables (which may differ from the actual lengths of these
parameter vectors). The probability mass function for the visible
variables \(\boldsymbol X = (X_1, \dots, X_N)\) follows from
marginalizing the joint specification to yield
\begin{equation}
\label{eq:RBM2}
P_{\boldsymbol \theta_N} (\boldsymbol x) = \sum\limits_{\boldsymbol h \in \{\pm 1\}^{N_{\mathcal{H}}}} \tilde{P}_{\boldsymbol \theta_N} (\boldsymbol x, \boldsymbol h), \qquad \boldsymbol x \in \{\pm 1\}^{N}\equiv \mathcal{X}^N.
\end{equation}
Here the baseline model \eqref{eq:RBM1} for hidden/visible variables is a
linear exponential one in sufficient statistics
\((\tilde{\boldsymbol X}, \boldsymbol X^T\boldsymbol H)\) using
\(\tilde{\boldsymbol X}=(\boldsymbol X,\boldsymbol H)\) from
\eqref{eq:RBM1}, but the form differs from the previous exponential models
in \eqref{eq:expo} in that the lengths of parameters
\(\boldsymbol \theta_N\) and statistics
\((\tilde{\boldsymbol X}, \boldsymbol X^T\boldsymbol H)\) increase to
incorporate more visible variables. That is, in contrast to
\eqref{eq:expo}, the natural parameter function involved in the RBM model
\eqref{eq:RBM1}, as the identity mapping of the parameters
\(\boldsymbol \theta_N\in\mathbb{R}^{q(N)}\), naturally grows in
dimension \(q(N)\to \infty\) to accommodate visible variables
\(X_1, \dots, X_N\) of increasing sample size \(N\to \infty\).
Additionally, one may further arbitrarily choose the number
\(N_\mathcal{H}\) of hidden variables \(\boldsymbol H\) in the joint RBM
model \eqref{eq:RBM1} to define a marginal model \eqref{eq:RBM2} for the
\(N\) visible variables \(\boldsymbol X\), and the number
\(N_\mathcal{H}\) of hiddens may also potentially increase with \(N\).
Because \(|\mathcal{X}| = 2\) and
\(P_{\boldsymbol \theta_N}(\boldsymbol x) > 0\) for all
\(\boldsymbol x \in \mathcal{X}^N\), the RBM specification \eqref{eq:RBM2}
for visibles \(\boldsymbol\) corresponds to a FOES model, while the
joint distribution \eqref{eq:RBM1} for \((\boldsymbol X, \boldsymbol H)\)
is also a FOES model. As this example also indicates, any model formed
by marginalizing a base FOES model class, such as the RBM joint
specification \eqref{eq:RBM1}, is again a FOES model.

\subsection{Deep Learning}\label{deep-learning}

Consider two models with ``deep architecture'' that contain multiple
hidden (or latent) layers in addition to a visible layer of data, namely
a deep Boltzmann machine (Salakhutdinov and Hinton
\protect\hyperlink{ref-salakhutdinov2009deep}{2009}) and a deep belief
network Hinton, Osindero, and Teh
(\protect\hyperlink{ref-hinton2006fast}{2006}){]}. Let \(M\) denote the
number of hidden layers included in the model and let
\(N_{(H,1)}, \dots, N_{(H,M)}\) denote the numbers of hidden variables
within each hidden layer. Then the random vector
\(\tilde{\boldsymbol X} = \{H^{(1)}_1, \dots, H^{(1)}_{N_{(H,1)}}, \dots, H^{(M)}_1, \dots, H^{(M)}_{N_{(H,M)}}, \boldsymbol X\}\)
collects both the hidden variables
\(\{ H_{i}^{(j)} : i=1,\ldots, N_{(H,j)}, j=1,\ldots,M\}\) and visible
variables \(\boldsymbol X =(X_1,\ldots,X_N)\) in a deep probabilistic
model. Each variable outcome will again lie in
\(\mathcal{X} = \{-1,1\}\).

\textbf{Deep Boltzmann machine (DBM).} The DBM class of models maintains
conditional independence within all layers in the model by stacking RBM
models and only allowing conditional dependence between neighboring
layers. The joint probability mass function for a DBM is \[
\tilde{P}_{\boldsymbol \theta_N} ( \tilde{\boldsymbol x} ) = \exp\left[ \sum\limits_{i = 1}^M\boldsymbol \alpha^{(i)T} \boldsymbol h^{(i)} + \boldsymbol \beta^T \boldsymbol x + \boldsymbol h^{(1)T} \Gamma^{(0)} \boldsymbol x + \sum\limits_{i = 1}^{M - 1} \boldsymbol h^{(i)T} \Gamma^{(i)} \boldsymbol h^{(i + 1)} - \psi(\boldsymbol \theta_N) \right],
\] for
\(\tilde{\boldsymbol x} = (\boldsymbol h^{(1)}, \dots, \boldsymbol h^{(M)}, \boldsymbol x) \in \mathcal{X}^{N_{(H,1)} + \cdots + N_{(H,M)} +N}\)
where \[
\psi(\boldsymbol \theta_N) = \log \sum\limits_{\tilde{\boldsymbol x} \in \mathcal{X}^{N_{(H,1)} + \cdots + N_{(H,M)} +N}} \exp\left[ \sum\limits_{i = 1}^M\boldsymbol \alpha^{(i)T} \boldsymbol h^{(i)} + \boldsymbol \beta^T \boldsymbol x + \boldsymbol h^{(1)T} \Gamma^{(0)} \boldsymbol x + \sum\limits_{i = 1}^{M - 1} \boldsymbol h^{(i)T} \Gamma^{(i)} \boldsymbol h^{(i + 1)}\right],
\] is the normalizing function for
\(\boldsymbol \theta_N = (\boldsymbol \alpha^{(1)}, \dots, \boldsymbol \alpha^{(M)}, \boldsymbol \beta,\Gamma^{(0)}, \dots, \Gamma^{(M - 1)}) \in \Theta_N \subset \mathbb{R}^{q(N)}\),
consisting of model parameters \(\boldsymbol \beta \in \mathbb{R}^N\),
\(\boldsymbol \alpha^{(i)} \in \mathbb{R}^{N_{(H,i)}}\),
\(i = 1, \dots, M\), along with a matrix \(\Gamma^{(0)}\) of dimension
\(N_{(H,1)} \times N\), and matrices \(\Gamma^{(i)}\) of dimension
\(N_{(H,i)} \times N_{(H,i+1)}\) for \(i = 1, \dots, M-1\). The combined
parameter vector \(\boldsymbol \theta_N\) has total length
\(q(N)= N_{(H,1)}+\cdots N_{(H,M)} + N + N_{(H,1)}*N+N_{H,2}*H_{(H,1)}+\cdots +N_{(H,M)}*H_{(H,M)-1}\).
The probability mass function for the visible random variables
\(X_1, \dots, X_N\) follows from this joint specification as \[
P_{\boldsymbol \theta_N} (\boldsymbol x) = \sum\limits_{(\boldsymbol h^{(1)}, \dots, \boldsymbol h^{(M)}) \in \mathcal{X}^{N_{(H,1)} + \cdots + N_{(H,M)}}} \tilde{P}_{\boldsymbol \theta_N} (\boldsymbol h^{(1)}, \dots, \boldsymbol h^{(M)}, \boldsymbol x) , \qquad \boldsymbol x \in \mathcal{X}^N.
\] Again like the RBM case, the DBM model specification is an example of
a FOES model.

\textbf{Deep belief network (DBN).} A DBN resembles a DBM in that there
are multiple layers of latent random variables stacked in a deep
architecture with no conditional dependence between layers. The
difference between the DBM and DBN models is that all but the last
stacked layer in a DBN are Bayesian networks (see Pearl
\protect\hyperlink{ref-pearl985bayesian}{1985}), rather than RBMs. A
Bayesian network is a probabilistic graphical model that defines
conditional dependence to be directed, rather than undirected (as with
the RBM). Thus for visibles \(X_1, \dots, X_N\) with support
\(\mathcal{X}^N, \mid \mathcal{X} \mid < \infty\), a DBN is also a FOES
model with \(q(N)\) the length of parameter vector is dependent on the
dimension of the visibles because
\(P_{\boldsymbol \theta_N}(\boldsymbol x)>0\) for all
\(\boldsymbol x \in\mathcal{X}^N\). Commonly, as in logistic belief nets
(Neal \protect\hyperlink{ref-neal1992connectionist}{1992}), a ``weight''
parameter is placed on each interaction between visibles,
\(X_1, \dots, X_N\), and the first layer of latent variables,
\(H^{(1)}_1, \dots, H^{(1)}_{N_{(H,1)}}\), satisfying the definition of
a FOES model.

\section{Main Results on Model Instability}\label{instability-results}

We now present a formal definition for instability of FOES models as
well as a simple condition for identifying instability in a FOES model
sequence.

\subsection{A Criterion for Instability}\label{criterion}

To define a measure of instability in FOES models, it is useful to
consider the behavior of data models \(P_{\theta_N}\), again supported
on a set \(\mathcal{X}^N\) of outcomes for
\(\boldsymbol X\equiv \boldsymbol X_N =(X_1,\ldots,X_N)\), in connection
to the sample size \(N\). A relevant quantity to this end is a log-ratio
of extremal probabilities (LREP), defined as
\begin{align}
\label{eq:elpr}
 \REP (\boldsymbol \theta_N)  =  \log \left[\frac{\max\limits_{  \boldsymbol x\in \mathcal{X}^N}P_{\boldsymbol \theta_N}( \boldsymbol x)}{\min\limits_{ \boldsymbol x \in \mathcal{X}^N}P_{\boldsymbol \theta_N}( \boldsymbol x)}\right],
\end{align}
based on maximum and minimal model probabilities. In what follows, the
main idea is that instability, and other negative model features, can be
associated with a FOES model formulation for \(N\) random variables
where the LREP \eqref{eq:elpr} is overly large relative to the sample size
\(N\). That is, a sequence of FOES probability models
\(P_{\boldsymbol \theta_N}\) results in specifying the distribution of
observations \(\boldsymbol X=(X_1,\ldots,X_N)\) for each sample size
\(N \geq 1\) and instability will generally occur among these models
whenever the corresponding LREP \eqref{eq:elpr} grows faster than \(N\).
This leads to the following definition.

\BeginKnitrBlock{definition}[S-unstable FOES model]
\protect\hypertarget{def:instabFSFS}{}{\label{def:instabFSFS}
\iffalse (S-unstable FOES model) \fi{} }A FOES model formulation for
\(\boldsymbol X_N=(X_1,\ldots,X_N)\) is \emph{Schweinberger-unstable} or
\emph{S-unstable} if
\begin{equation}
\label{eq:Sun}
\lim \limits_{N \rightarrow \infty} \frac{1}{N} \REP(\boldsymbol \theta_N) \equiv \lim \limits_{N \rightarrow \infty} \frac{1}{N}\log \left[\frac{\max\limits_{  \boldsymbol x\in \mathcal{X}^N}P_{\boldsymbol \theta_N}( \boldsymbol x)}{\min\limits_{ \boldsymbol x \in \mathcal{X}^N}P_{\boldsymbol \theta_N}( \boldsymbol x)}\right] = \infty
\end{equation}
as the number of variables increases (\(N \rightarrow \infty\)).
\EndKnitrBlock{definition} In other words, a model is S-unstable if
\(\REP(\boldsymbol \theta_N)/N\) is an unbounded sequence of sample size
\(N\); namely, given any \(C > 0\), there exists an integer \(N_C > 0\)
so that \(\REP(\boldsymbol \theta_N)/N > C\) holds for all
\(N \ge N_C\). A FOES model formulation may be termed S-stable if it
fails to be S-unstable, i.e., if
\(\sup_{N \geq 1}\REP(\boldsymbol \theta_N)/N\) is bounded.

This definition of S-unstable is a generalization or reinterpretation of
``unstable'' used in Schweinberger
(\protect\hyperlink{ref-schweinberger2011instability}{2011}) by allowing
possibly non-exponential family models (e.g., RBM and DBM models in
Sections \ref{rbm}-\ref{deep-learning} as well as a potentially
increasing number \(q(N)\) of parameters through the parameter sequence
\(\boldsymbol \theta_N\in \mathbb{R}^{q(N)}\). While this definition
differs in form and scope from the original, it does match that in
Schweinberger
(\protect\hyperlink{ref-schweinberger2011instability}{2011}) for the
special case of exponential models (cf.~Section
\ref{discrete-exponential-family-models} considered there. Section
\ref{illustrations} provides several examples of unstable models as well
as causes for model instability, where the latter may often be traced to
issues in model form (i.e., data functions) and/or parameterization. We
next describe several potentially undesirable features associated with
S-unstable FOES models.

\BeginKnitrBlock{remark}
\iffalse{} {Remark. } \fi{}In the definition \eqref{eq:Sun} of
S-instability, we note that the numerical measure
\(\REP(\boldsymbol \theta_N)/N\) of model instability is invariant to
\emph{independent replications} of data. That is, let \(M \geq 1\)
denote a possible number of replications and consider data
\(\boldsymbol Y_{N,M} \equiv (\boldsymbol X^{(1)}_N, \dots, \boldsymbol X^{(M)}_N)\)
formed by \(\{ \boldsymbol X^{(j)}_N\}_{j=1}^M\) as \(M\) iid
replications of a random vector \(\boldsymbol X_N=(X_1,\ldots,X_N)\),
where the latter follows a FOES model with probabilities
\(P_{ \boldsymbol \theta_N}(\boldsymbol x)>0\),
\(\boldsymbol x\in\mathcal{X}^N\). This leads to a joint model, say
\(P_{ \boldsymbol \theta_N}(\boldsymbol y)\),
\(\boldsymbol y\in\mathcal{X}^{NM}\), for \(\boldsymbol Y_{N,M}\)
consisting of \(N*M\) random variables in total. Then, the LREP for
\(\boldsymbol Y_{N,M}\), scaled by associated size, is given by
\begin{align*}
\frac{1}{NM}\REP_{\boldsymbol Y_{N,M}}( \boldsymbol\theta_N ) &\equiv \frac{1}{NM}\log\left[\frac{\max_{\boldsymbol y \in \mathcal{X}^{NM}}P_{ \boldsymbol \theta_N}(\boldsymbol y)  }{\min_{\boldsymbol y \in \mathcal{X}^{NM}}P_{ \boldsymbol \theta_N}(\boldsymbol y)} \right] \\
&= \frac{1}{NM}\log\left[\frac{\max_{\boldsymbol x \in \mathcal{X}^{N}}P_{ \boldsymbol \theta_N}(\boldsymbol x)  }{\min_{\boldsymbol x \in \mathcal{X}^{N}}P_{ \boldsymbol \theta_N}(\boldsymbol x)} \right]^M \equiv \frac{1}{N}\REP_{\boldsymbol X_{N}}( \boldsymbol\theta_N ),
\end{align*}
where
\(\REP_{\boldsymbol X_{N}}( \boldsymbol\theta_N ) \equiv \REP( \boldsymbol\theta_N )\)
denotes the log-ratio of extremal probabilities for \(\boldsymbol X_N\)
defined from \eqref{eq:elpr}. That is, due to iid properties, the
sample-size corrected LREP for \(\boldsymbol Y_{N,M}\) equals the
analog, \(\REP(\boldsymbol\theta_N )/N\), from the underlying common
data model for \(\boldsymbol X_N\) alone, regardless of the level
\(M \geq 1\) of independent replication. Consequently, the definition of
an S-unstable model is unaffected by independent replication and all
instability properties may be characterized by those of one observation
from the common FOES model. For computational purposes, this aspect also
implies that if the original data \(\boldsymbol X_N=(X_1,\ldots,X_N)\)
in a FOES model consist of \(N\) iid random variables, then the
size-scaled log-ratio \eqref{eq:elpr} may be calculated as \[
\frac{1}{N}\REP( \boldsymbol\theta_N ) \equiv \frac{1}{N}\REP_{\boldsymbol X_{N}}( \boldsymbol\theta_N ) = \log\left[ \frac{\max_{ x \in \mathcal{X}}P_{\boldsymbol\theta_N}(X_1=x)} {\min_{ x \in \mathcal{X}}P_{\boldsymbol\theta_N}(X_1=x)} \right]
\] based on the extremal probabilities of just one random variable
\(X_1\).
\EndKnitrBlock{remark}

\subsection{Characterizations and Consequences of
Instability}\label{characterizations-and-consequences-of-instability}

As a basic characteristic, S-unstable FOES model sequences have
extremely sensitive probability structures. One aspect is that small
changes in data configuration can lead to very large changes in
probability. Consider, for example, the quantity given by \[
\DN(\boldsymbol \theta_N) \equiv \max \left\{\log \frac{P_{\boldsymbol \theta_N}(\boldsymbol x)}{P_{\boldsymbol \theta_N}(\boldsymbol x^*)} : \boldsymbol x \text{ }\& \text{ } \boldsymbol x^* \in \mathcal{X}^N \text{ differ in exactly one component}\right\},
\] which represents the biggest log-probability ratio for a one
component change in data outcomes in a FOES model with parameter
\(\boldsymbol \theta_N\). We then have the following result prescribing
the behavior of \(\DN(\boldsymbol \theta_N)\) for S-unstable FOES
models.

\BeginKnitrBlock{theorem}
\protect\hypertarget{thm:instab-elpr}{}{\label{thm:instab-elpr}}Let
\(P_{\boldsymbol \theta_N}\), with support \(\mathcal{X}^N\),
\(N\geq 1\), be a sequence of FOES models.
\begin{enumerate}[(i)]
\item For any integer $N \geq 1$ and any given $C>0$, if $\REP(\boldsymbol \theta_N)/N > C$ in (\ref{eq:elpr}, then
    $$ \DN(\boldsymbol \theta_N) > C,$$
    or probabilities from a one component change in some outcome have log-ratio exceeding $C$.
\item Suppose the FOES model sequence is S-unstable. Then, for all large $N$ and given any arbitrary $C>0$, there exist outcomes $\boldsymbol x,\boldsymbol x^*\in\mathcal{X}^N$, differing by one component, such that
    $$
    \frac{P_{\boldsymbol \theta_N}(\boldsymbol x)}{P_{\boldsymbol \theta_N}(\boldsymbol x^*)} > \exp[N C].
    $$
\end{enumerate}
\EndKnitrBlock{theorem}

Theorem \ref{thm:instab-elpr}(i) is a non-asymptotic result, which
connects to the definition of instability in a FOES model through a
log-ratio of extreme model probabilities \eqref{eq:elpr} being too large
relative to the associated sample size \(N\). If so, Theorem
\ref{thm:instab-elpr}(i) guarantees the FOES model must also exhibit
correspondingly large changes in probability for very small differences
among some data configurations, a property that intuitively captures a
notion of instability. Furthermore, and perhaps more seriously under
Theorem \ref{thm:instab-elpr}(ii), S-unstable models can never have
universally bounded changes in probability among single component
variations in data configurations. While not all one component changes
in data may produce massive changes in probability, unstable models must
have some such data outcomes with this property. As a consequence,
unstable probability structures may exhibit extreme sensitivity through
large peaks and troughs over the sample space.

Additionally, S-unstable FOES model sequences are also connected to
degenerate models, where \emph{degeneracy} involves assigning
essentially all probability to modes within the sample space, which
could potentially represent a small subset among the totality of
outcomes. For perspective, note that differing sizes of the scaled
log-ratio \(\REP(\boldsymbol \theta_N)/N\) from \eqref{eq:elpr} induce a
spectrum of levels of instability/stability and Theorem
\ref{thm:instab-elpr} indicates increasing sensitivity of model
probabilities as \eqref{eq:elpr} increases. Furthermore, as the
instability measure grows and the log-ratio
\(\REP(\boldsymbol \theta_N)/N\) diverges, as in the definition
\eqref{eq:Sun} of S-unstable models, then a FOES model sequence will
become degenerate. Theorem \ref{thm:degenFOES} provides a formal
statement of such degeneracy due to S-instability. For a given
\(0 < \epsilon < 1\), define a \(\epsilon\)-modal set of outcomes as
\begin{equation}
\label{eq:mode}
\mathcal{M}_{\epsilon, \boldsymbol \theta_N} \equiv \left\{\boldsymbol x \in \mathcal{X}^N: \log P_{\boldsymbol \theta_N}(\boldsymbol x) > (1-\epsilon)\max\limits_{\boldsymbol y \in \mathcal{X}^N} \log  P_{\boldsymbol \theta_N}(\boldsymbol y) + \epsilon\min\limits_{\boldsymbol y \in \mathcal{X}^N} \log P_{\boldsymbol \theta_N}(\boldsymbol y) \right\}.
\end{equation}
\BeginKnitrBlock{theorem}
\protect\hypertarget{thm:degenFOES}{}{\label{thm:degenFOES}}For any
arbitrarily small \(0 < \epsilon < 1\), an S-unstable FOES model
sequence \(P_{\boldsymbol \theta_N}\), \(N \geq 1\), for
\(\boldsymbol X_N=(X_1, \dots, X_N)\) satisfies \[
P_{\boldsymbol \theta_N}\left( \boldsymbol X_N\in \mathcal{M}_{\epsilon, \boldsymbol \theta_N}\right) \rightarrow 1 \text{ as } N \rightarrow \infty.
\]
\EndKnitrBlock{theorem}

In other words, as the sample size grows in S-unstable FOES models, all
probability tends to concentrate mass on an \(\epsilon\)-modal set,
where \(\epsilon\) can be made arbitrarily small. Intuitively, the
occurrence of such degeneracy can be explained by a type of ``reverse''
pigeonhole principle for unstable FOES models: if all outcomes should
receive positive probability but the maximal probability far exceeds the
minimal one in the model, then little probability remains for
distribution among remaining model outcomes (i.e., if nearly all
available pigeons are stuffed into one hole, the remaining pigeonholes
must have few occupants). Degeneracy in unstable models can pose dangers
in data modeling as well, particularly when a mode set represents a
narrow collection of outcomes among those realistically possible for
adequately describing data. In which case, model outcomes may fail to
look like data of interest.

Connected to degeneracy, S-unstable FOES models may also exhibit
additional kinds of extreme and undesirable sensitivity in probabilities
if model parameters \(\boldsymbol \theta_N\) can further be ``dialed''
between positive and negative values. That is, some FOES models
naturally involve parameter spaces covering a positive-negative spectrum
of parameter possibilities, where the signs of parameters provide a
standard device for increasing or decreasing probabilities of outcomes
in the model formulation. In fact, for many models, the switch of a
parameter sign serves to produce reciprocal probabilities, as outlined
in the following model assumption about parameter sign reversal (PSR).

\textbf{Model Condition PSR} \emph{(Reciprocal Probabilities from
Parameter Sign Reversal)}: Let \(P_{\boldsymbol \theta_N}\), with
support \(\mathcal{X}^N\), \(N\geq 1\), represent a sequence of FOES
models. For each \(N \geq 1\) and any outcome
\(\boldsymbol x \in \mathcal{X}^N\), suppose it holds that \[
P_{\boldsymbol \theta_N}(\boldsymbol x)  \cdot P_{-\boldsymbol \theta_N}(\boldsymbol x) \;=\;   \max\limits_{\boldsymbol y \in \mathcal{X}^N}P_{ \boldsymbol \theta_N}(\boldsymbol y)\cdot \min\limits_{\boldsymbol y \in \mathcal{X}^N}P_{-\boldsymbol \theta_N}(\boldsymbol y),
\] where
\(\max_{\boldsymbol y \in \mathcal{X}^N}P_{ \boldsymbol \theta_N}(\boldsymbol y)\)
and
\(\min_{\boldsymbol y \in \mathcal{X}^N}P_{-\boldsymbol \theta_N}(\boldsymbol y)\)
denote the maximum and minimum probabilities under parameters
\(\boldsymbol \theta_N\) and \(-\boldsymbol \theta_N\), respectively.

The above model condition incorporates many standard parameterizations
and follows, for instance, whenever
\(P_{\boldsymbol \theta_N}(\boldsymbol x)/P_{\boldsymbol \theta_N}(\boldsymbol y) = [P_{-\boldsymbol \theta_N}(\boldsymbol x)/P_{-\boldsymbol \theta_N}(\boldsymbol y)]^{-1}\)
holds for outcomes \(\boldsymbol x, \boldsymbol y \in\mathcal{X}^N\) in
a FOES model. For instance, this latter condition is fulfilled for all
linear exponential families from Section
\ref{discrete-exponential-family-models} (e.g.,
\eqref{eq:mod1}-\eqref{eq:mod2}) as well as all network models from Sections
\ref{rbm}-\ref{deep-learning} (e.g., \eqref{eq:RBM1}-\eqref{eq:RBM2}). When
parameters can be tuned in sign with effects prescribed in the model
condition PSR, unstable FOES models will exhibit further probability
sensitivities, as outlined in the following extension of Theorem
\ref{thm:degenFOES}.

\BeginKnitrBlock{corollary}
\protect\hypertarget{cor:sign}{}{\label{cor:sign} }Let
\(P_{\boldsymbol \theta_N}\), with support \(\mathcal{X}^N\),
\(N\geq 1\), be a sequence of FOES models satisfying model condition
PSR. If the models \(P_{\boldsymbol \theta_N}\) are additionally
S-unstable, then
\begin{enumerate}[(i)]
\item the models  $P_{-\boldsymbol \theta_N}$  defined by $-\boldsymbol \theta_N$  are also S-unstable;
\item and for the complement $\mathcal{M}_{\epsilon, \boldsymbol \theta_N}^c \equiv \mathcal{X}^N \setminus \mathcal{M}_{\epsilon, \boldsymbol \theta_N}$ of any mode-set $\mathcal{M}_{\epsilon, \boldsymbol \theta_N}$ under $\boldsymbol \theta_N$ from (\ref{eq:mode}), with $0<\epsilon<1$, it holds under $-\boldsymbol \theta_N$ that
    $$ 
    P_{-\boldsymbol \theta_N}( \boldsymbol X_N \in \mathcal{M}_{\epsilon, \boldsymbol \theta_N}^c) \rightarrow 1\qquad \text{as } N\to \infty,
    $$
    while, by Theorem \ref{thm:degenFOES}, $P_{\boldsymbol \theta_N}( \boldsymbol X_N \in \mathcal{M}_{\epsilon, \boldsymbol \theta_N} ) \rightarrow 1$ holds for $\boldsymbol X_N = (X_1,\ldots,X_N)$ under $\boldsymbol \theta_N$.
\end{enumerate}
\EndKnitrBlock{corollary}

For unstable models, Corollary \ref{cor:sign} shows that shifts in
parameters around zero (i.e., from \(\boldsymbol \theta_N\) to
\(-\boldsymbol \theta_N\)) can induce extreme changes in probability
among subsets of the sample space, as another manifestation of
instability and hyper-sensitivity in probability structure. For
one-parameter exponential families, involving a fixed real-valued linear
parameter \(\boldsymbol \theta_N = \theta \in \mathbb{R}\) and
sufficient statistic \(\boldsymbol g_N(\boldsymbol x)\in \mathbb{R}\) in
\eqref{eq:expo}, Schweinberger
(\protect\hyperlink{ref-schweinberger2011instability}{2011} Theorem 3)
proved a result similar in spirit, though based on a characterization
there in terms of maximum
\(U_N \equiv \max_{\boldsymbol x\in\mathcal{X}^N}g_N(\boldsymbol x)\)
and minimal
\(L_N \equiv \min_{\boldsymbol x\in\mathcal{X}^N}g_N(\boldsymbol x)\)
values of the sufficient statistic. For this case in particular, mode
sets have specific, and essentially complementary, forms over positive
and negative parameters, namely,
\(\mathcal{M}_{\epsilon, \boldsymbol \theta_N} = \{\boldsymbol x \in\mathcal{X}^N: g_N(\boldsymbol x) > (1-\epsilon) U_N + \epsilon L_N \}\)
and
\(\mathcal{M}_{\epsilon, -\boldsymbol \theta_N} = \{\boldsymbol x \in\mathcal{X}^N: g_N(\boldsymbol x) < \epsilon U_N + (1-\epsilon) L_N \}\)
for any \(\boldsymbol \theta_N>0\), and Schweinberger
(\protect\hyperlink{ref-schweinberger2011instability}{2011} Theorem 3)
showed each mode set collects all mass, under positive and negative
parameters, respectively, with unstable models of this exponential type.
However, for all unstable FOES models, Corollary \ref{cor:sign}
generalizes the same principle that unstable models can push all
probability to different, and in fact disjoint, parts of the sample
space, depending on how parameters fall with respect to zero. This
feature can numerically complicate likelihood manipulations, such as
maximization or MCMC-based Bayes posterior sampling, as further
discussed in Section \ref{implications}.

\BeginKnitrBlock{remark}
\iffalse{} {Remark. } \fi{}Under the model condition PSR, Corollary
\ref{cor:sign} can also be extended to cases where parameter components
\(\boldsymbol \theta_N =(\boldsymbol \theta_{1,N}, \boldsymbol \theta_{N,2})\)
(say) are not all changed in sign (e.g., \(-\boldsymbol \theta_N\)) but,
more generally, are instead altered to another parameter configuration
\(\boldsymbol \theta_{N}^A = ( \boldsymbol \theta_{1,N}^A, \boldsymbol \theta_{2,N}^A )\)
involving a switch in sign only among some dominating model parameters
\(\boldsymbol \theta_{N,2}^A=- \boldsymbol \theta_{N,2}\) with remaining
parameters \(\boldsymbol \theta_{1,N}^A\) being arbitrarily chosen. If a
change sign occurs among parameters (\(\pm \boldsymbol \theta_{N,2}\))
which dominate the probability structure of the model, then the results
of Corollary \ref{cor:sign} can still hold with
\(\boldsymbol \theta_{N}^A\) replacing \(-\boldsymbol \theta_N\); as an
example of one sufficient condition, if
\(\lim_{N\to \infty} \max_{\boldsymbol x \in \mathcal{X}^N} |G_N(\boldsymbol x, -\boldsymbol \theta_{N}) - G_N(\boldsymbol x, \boldsymbol \theta_{N}^A)|=0\)
holds in addition to Corollary \ref{cor:sign} assumptions, where \[
G_N(\boldsymbol x, \boldsymbol \theta) =   \frac{\log P_{\boldsymbol \theta}(\boldsymbol x) - \min_{\boldsymbol y \in\mathcal{X}^N}\log P_{\boldsymbol \theta}(\boldsymbol y) }{\max_{\boldsymbol y \in\mathcal{X}^N}\log P_{\boldsymbol \theta}(\boldsymbol y)  - \min_{\boldsymbol y \in\mathcal{X}^N}\log P_{\boldsymbol \theta}(\boldsymbol y) }, \quad \boldsymbol x \in\mathcal{X}^N,
\] represents a standardized form of \(\boldsymbol \theta\)-model
probabilities, then the results of Corollary \ref{cor:sign} apply to
\(\boldsymbol \theta_{N}^A\) in addition to \(-\boldsymbol \theta_N\).
As a consequence, an unstable model under \(\boldsymbol \theta_N\) can
then imply that many more unstable models exist over a broader spectrum
of possibilities for variations \(\boldsymbol \theta_{N}^A\) of
\(\boldsymbol \theta_N\), which involves some amount of sign change
among components of \(\boldsymbol \theta_N\).
\EndKnitrBlock{remark}

\section{Illustrations}\label{illustrations}

Model instability can depend intricately on how functions of parameters
and data \(\boldsymbol X_N=(X_1,\ldots,X_n)\) are combined in the
formulation of the model probabilities, though some general causes may
be identified. As one issue, a broad parameter space (or wide
interpretation of this space) may admit some parameters as technically
valid that have an undue and often undesirable impact on the model
structure for a prescribed data size \(N\). In this case, both the size
and dimension of model parameters can be problematic and induce
instability. In combination to this last point, further causes of
instability may also be traced to the magnitude of statistics in the
model. Potentially massive, and thereby unstable, statistics were the
primary focus of instability studies of Schweinberger
(\protect\hyperlink{ref-schweinberger2011instability}{2011}) for certain
discrete exponential models having parameters/statistics of fixed
dimension. However, as shown in the following, bounded statistics may
still lead to instability if the parameter dimension is high. We next
provides some examples to illustrate S-instability in FOES models, which
also suggest some potential strategies for preventing unstable models.

\subsection{Equi-probability Models}\label{equi-probability-models}

As a baseline for comparisons, consider a simplistic model for
\(\boldsymbol X_N=(X_1,\ldots,X_N)\) with uniform probabilities over the
sample space, say
\(P_{\boldsymbol \theta_N}(\boldsymbol x)= |\mathcal{X}|^{-N}\),
\(\boldsymbol x \in \mathcal{X}^N\), where each random variable has
\(|\mathcal{X}| \geq 1\) outcomes. In contrast to instability, model
probabilities here are completely insensitive to changes in data
outcomes across the sample space, and the associated log-ratio of
extreme probabilities \eqref{eq:elpr} is \[
\frac{1}{N} \REP(\boldsymbol \theta_N)=0\quad \;(\text{uniform probability model}),
\] which is as small as possible. In fact, a LREP value of zero can only
occur for a FOES model having uniform probabilities, and such
equi-probability models are always S-stable.

\subsection{One-parameter Exponential Models}\label{one-param-exp}

A fundamental model considered in the instability work of Schweinberger
(\protect\hyperlink{ref-schweinberger2011instability}{2011}) involves a
one-parameter exponential model corresponding to \eqref{eq:expo} with a
real-valued parameter, say
\(\boldsymbol \theta_N = \eta(\boldsymbol \theta_N)\in \mathbb{R}\), and
sufficient statistic \(\boldsymbol g_N(\boldsymbol x)\in \mathbb{R}\).
For such models, upon scaling by sample size \(N\), the log-ratio of
extreme probabilities in \eqref{eq:elpr} for assessing instability becomes
\begin{equation}
\label{eq:UL}
\frac{1}{N}\REP(\boldsymbol \theta_N ) \equiv   |\boldsymbol \theta_N| \frac{(U_N-L_N)}{N} \;\quad \text{(one-parameter exponential model)},
\end{equation}
where
\(U_N \equiv \max_{\boldsymbol x\in\mathcal{X}^N}g_N(\boldsymbol x)\)
and
\(L_N \equiv \min_{\boldsymbol x\in\mathcal{X}^N}g_N(\boldsymbol x)\)
denote the maximal and minimal values of the single sufficient
statistic. In this case, an S-unstable model results, by definition
\eqref{eq:Sun}, whenever
\(\lim_{N\to \infty} |\boldsymbol \theta_N| (U_N-L_N)/N= \infty\) holds
or, in other words, if the combined magnitudes of parameter
\(|\boldsymbol \theta_N|\) and maximal difference \(U_N-L_N\) in
statistic values are overwhelmingly large relative to the sample size
\(N\). If we further assume that
\(\boldsymbol \theta_N =\theta\in\mathbb{R}\setminus \{0\}\) is a fixed
(non-zero) parameter for all \(N \geq 1\), as considered in
Schweinberger
(\protect\hyperlink{ref-schweinberger2011instability}{2011}), then an
S-unstable model results solely if the sufficient statistic admits a
value \(U_N-L_N\) too large relative to number \(N\) of observations,
i.e., if \((U_N-L_N)/N\to \infty\) as \(N\to \infty\). The latter aspect
reflects the definition of Schweinberger
(\protect\hyperlink{ref-schweinberger2011instability}{2011}), for this
setting, that a real-valued \emph{statistic} \(g_N(\boldsymbol x)\) may
be classified as \emph{unstable} when
\(\lim_{N\to \infty}|(U_N-L_N)/N=\infty\) holds and as \emph{stable}
otherwise (e.g., if \(\sup_{N \geq 1}(U_N-L_N)/N<\infty\)).

For illustration, consider the iid Bernoulli model \eqref{eq:mod1} for
\(\boldsymbol X_N=(X_1,\ldots,X_N)\) with log-odds ratio parameter
\(\boldsymbol \theta_N = \log[ P_{\boldsymbol \theta_N}(X_1=1)/ P_{\boldsymbol \theta_N}(X_1=0)]\in\mathbb{R}\).
Remark 1 (Section \ref{criterion}) then gives the model instability
measure \eqref{eq:Sun} directly as \[
\frac{1}{N}\REP(\boldsymbol \theta_N ) = |\boldsymbol \theta_N|\quad\; \text{(iid Bernoulli model)},
\] so that an unstable (or stable) model results for a divergent (or
bounded) parameter sequence \(|\boldsymbol \theta_N|\). The above
instability expression for the Bernoulli model follows as well from the
\(N\)-scaled LREP value \eqref{eq:UL} for a one-parameter exponential
distribution, using that the sufficient statistic involved
\(g_N(\boldsymbol x)= \sum_{i=1}^N x_i\),
\(\boldsymbol x =(x_1,\ldots,x_n)\in\{0,1\}^N\), has maximum and minimum
values \(U_N=N\) and \(L_N=0\). In this case, Schweinberger
(\protect\hyperlink{ref-schweinberger2011instability}{2011}) has noted
that the statistic is stable (i.e., bounded \((U_N-L_N)/N=1\)) and the
Bernoulli model is as well when, in particular,
\(\boldsymbol \theta_N=\bold \theta \in\mathbb{R}\) is fixed for
\(N \geq 1\).

Alternatively, considering a random graph with \(N={n \choose 2}\) edges
among \(n\) nodes, the exponential graph model from \eqref{eq:mod2}, when
based purely on the number of \(g_{2,N}(\boldsymbol x)\) of 2-stars or
solely the number \(g_{3,N}(\boldsymbol x)\) of triangles,
\(\boldsymbol x\in\{0,1\}^N\), has an measure of instability from
\eqref{eq:Sun} as \[
\frac{1}{N}\REP(\boldsymbol \theta_N )  = \left\{ \begin{array}{lcl} |\boldsymbol \theta_N| (n-2) && \text{(2-star graph model)}\\
|\boldsymbol \theta_N|(n-2)/3 &&\text{(triangle graph model)},\end{array}\right.
\] by using the (one-parameter exponential) LREP formula \eqref{eq:UL}
with statistic maximums \(U_N= N(n-2)\) for 2-stars or \(U_N= N(n-2)/3\)
for triangles and with minimums \(L_N=0\) in both cases. Because the
variable number \(N\to \infty\) as the node number \(n\to \infty\), both
counts of 2-stars and triangles are unstable statistics in the sense of
Schweinberger
(\protect\hyperlink{ref-schweinberger2011instability}{2011}) (i.e.,
\(\lim_{N\to \infty} (U_N-L_N)/N=\infty\)). Furthermore, both types of
graph models are always S-unstable for all possible of parameter
sequences \(\boldsymbol \theta_N \in\mathbb{R}\) that are bounded away
from zero (i.e.,
\(\lim_{N\to \infty}\REP(\boldsymbol \theta_N )/N=\infty\) then holds,
including the fixed parameter case
\(\boldsymbol \theta_N=\theta\in\mathbb{R}\setminus \{0\}\) from
Schweinberger
(\protect\hyperlink{ref-schweinberger2011instability}{2011})).

\subsection{Fixed-dimensional Linear Exponential
Models}\label{fixed-dim-exp}

As a generalization of the one-parameter exponential case, we next
consider linear exponential families \eqref{eq:expo} with \(k\) parameters
\(\boldsymbol \theta_N = (\theta_{1,N},\ldots,\theta_{k,N})^\prime\) and
\(k\) sufficient statistics
\(\boldsymbol g_N(\boldsymbol x) = (g_{1,N}(\boldsymbol x),\ldots, g_{k,N}(\boldsymbol x))^\prime\).
Here the dimension \(k\) of model parameters/statistics is fixed, and we
next prescribe a condition helpful to avoiding instability in such
models. For this, define
\(U_{i,N}=\max_{\boldsymbol x \in\mathcal{X}^N} g_{i,N}(\boldsymbol x)\)
and
\(L_{i,N}=\min_{\boldsymbol x \in\mathcal{X}^N} g_{i,N}(\boldsymbol x)\)
as the maximal and minimal values of the \(i\)th statistic,
\(i=1,\ldots,k\), based on observations
\(\boldsymbol X_N=(X_1,\ldots,X_N)\).

\BeginKnitrBlock{proposition}
\protect\hypertarget{prp:prop1}{}{\label{prp:prop1}}Let
\(P_{\boldsymbol \theta_N}\), \(N \geq 1\), denote linear exponential
models (\ref{eq:expo}) with parameters
\(\boldsymbol \theta_N = (\theta_{1,N},\ldots,\theta_{k,N})^\prime \in \mathbb{R}^k\)
and statistics
\(\boldsymbol g_N(\boldsymbol x) = (g_{1,N}(\boldsymbol x),\ldots, g_{k,N}(\boldsymbol x))^\prime \in \mathbb{R}^k\),
for fixed \(k \geq 1\). Then, the models \(P_{\boldsymbol \theta_N}\)
are \{\rm S-stable\} if
\begin{equation}
\label{eq:prop1}
\sup_{N \geq 1}\frac{1}{N} \max_{1 \leq i \leq k }|\theta_{i,N}|(U_{i,N}-L_{i,N})<\infty
\end{equation}
holds, i.e., if
\(\max_{1 \leq i \leq k } |\theta_{i,N}|(U_{i,N}-L_{i,N})/N\) is bounded
sequence of sample size \(N\).
\EndKnitrBlock{proposition}

\BeginKnitrBlock{remark}
\iffalse <span class="remark"><em>Remark. \fi{} In the one-parameter
exponential case \(k=1\), recall the exponential model is
stable/unstable depending on whether
\(\REP(\boldsymbol \theta_N)/N = |\theta_{1,N}|(U_{1,N}-L_{1,N})/N \equiv |\boldsymbol \theta_{N}|(U_{N}-L_{N})/N\)
in \eqref{eq:UL} is convergent/divergent. Hence, for \(k=1\), the
condition \eqref{eq:prop1} of Proposition \ref{prp:prop1} captures the
same notion of S-stability based on \eqref{eq:UL}.
\EndKnitrBlock{remark}

Proposition \ref{prp:prop1} provides a sufficient condition for the
stability of linear exponential models with fixed parameter dimension
\(k\geq 1\), whereby an S-stable model is guaranteed if the compounded
magnitude of each combination of parameter \(\theta_{i,N}\) and
sufficient statistic value \((U_{i,N}-L_{i,N})\) is bounded by the
sample size \(N\), \(i=1,\ldots,k\). This supports the findings of
Schweinberger
(\protect\hyperlink{ref-schweinberger2011instability}{2011}), who showed
degeneracy follows in such models under one type of violation of the
condition \eqref{eq:prop1} in Proposition \ref{prp:prop1} (namely,
involving \(k>1\) non-zero parameters with \(k-1\) statistics being
\(O(N)\) bounded while one statistic diverges in maximal size faster
than the number \(N\) of observations). To further illustrate the result
in Proposition \ref{prp:prop1}, consider the multinomial distribution
\eqref{eq:mod11} for \(\boldsymbol X_N=(X_1,\ldots,X_N)\) having
\(k\geq 2\) categories \(\{1,\ldots,k\}\) and \(k\) parameters
\(\boldsymbol \theta_N = (\theta_{1,N},\ldots,\theta_{k,N})^\prime\).
The variables are iid under this model so that Remark 1 (Section
\ref{criterion} yields the corresponding \(N\)-scaled log-ratio of
extreme probabilities \eqref{eq:elpr} as
\begin{align*}
  \frac{1}{N}\REP(\boldsymbol \theta_N) &= \frac{\max_{1 \leq i \leq k} P_{\boldsymbol \theta_N}(X_1=i)}{\min_{1 \leq i \leq k} P_{\boldsymbol \theta_N}(X_1=i)}\\
  &= \max_{1 \leq i \leq k} \theta_{i,N} - \min_{1 \leq i \leq k} \theta_{i,N} \qquad \text{(iid multinomial model)}.
\end{align*}
Hence, a multinomial model sequence is unstable (or stable) depending on
whether (or not) the maximal parameter difference
\(\max_{1 \leq i \leq k} \theta_{i,N} - \min_{1 \leq i \leq k} \theta_{i,N}\)
diverges. Furthermore, using that each of the \(k\) sufficient (count)
statistics from the multinomial model \eqref{eq:mod11} satisfies
\((U_{i,N}-L_{i,N})/N=1\), we see that \eqref{eq:prop1} of Proposition
\ref{prp:prop1} becomes purely a parameter condition,
\(\sup_{N \geq 1}\max_{1 \leq i \leq k } |\theta_{i,N}| <\infty\), for
ensuring that
\(\REP(\boldsymbol \theta_N)/N =\max_{1 \leq i \leq k} \theta_{i,N} - \min_{1 \leq i \leq k} \theta_{i,N}\)
is bounded and stability follows for the multinomial distribution.
Additionally, a stable multinomial sequence (i.e., bounded
\(\REP(\boldsymbol \theta_N)/N\)) turns out to be nearly equivalent to
\eqref{eq:prop1} (e.g., these are the same if the smallest parameter
\(\min_{1 \leq i \leq k } |\theta_{i,N}|\) remains bounded).

When the condition \eqref{eq:prop1} of Proposition \ref{prp:prop1} is
violated, this aspect suggests a potentially unstable model that may be
investigated more closely. For example, consider the exponential graph
model from \eqref{eq:mod2} involving counts of edges, 2-stars and
triangles with fixed parameters
\(\boldsymbol \theta_N = (\theta_{1},\theta_2,\theta_3)^\prime \in \mathbb{R}^3\)
for \(N\geq 1\). If either the 2-star parameter \(\theta_2 \neq 0\) or
triangle parameter \(\theta_3 \neq 0\) is non-zero, then
\(\max_{1 \leq i \leq 3 } |\theta_{i}|(U_{i,N}-L_{i,N})/N \propto (n-2)\to \infty\)
holds in \eqref{eq:prop1} by
\((U_{2,N}-L_{2,N})/N = 3 (U_{3,N}-L_{3,N})/N=(n-2)\) for 2-star and
triangle statistics (\(i=2,3\)), so that Proposition \ref{prp:prop1}
hints that an unstable model may result when
\(|\theta_2| + |\theta_3| \neq 0\). Relatedly, a result from
Schweinberger
(\protect\hyperlink{ref-schweinberger2011instability}{2011} Result 3)
states that this model is unstable for all fixed parameters excluding
cases \(\theta_2 =\theta_3=0\) or \(\theta_2 = - \theta_3/3\). However,
more is true in line with the instability suggested by Proposition
\ref{prp:prop1} whenever \(|\theta_2| + |\theta_3| \neq 0\) (i.e.,
excluding \(\theta_2 =\theta_3=0\)).

To see this, consider an even number \(n>2\) of nodes and let
\(\boldsymbol x_0\) denote the data outcome in
\(\mathcal{X}^N \equiv \{0,1\}^N\) with all \(N = {n \choose 2}\) edges
being zero, let \(\boldsymbol x_1\) denote the outcome with all edges
being 1, and let \(\boldsymbol x_2\) denote the edge configuration from
dividing the nodes into two equal groups, with no edges within a group
and all edges between the groups (so that no triangles exist in
\(\boldsymbol x_2\)). Then, the \(N\)-scaled log-ratio \eqref{eq:elpr} for
the exponential graph model \eqref{eq:mod2} can, by definition, be bounded
below by
\begin{align*}
\frac{1}{N}\REP_N(\boldsymbol \theta_N) &\geq \max_{i=1,2}\frac{1}{N}
\left| \log\left[ \frac{P_{\boldsymbol \theta_N}(\boldsymbol x_i)}{P_{\boldsymbol \theta_N}(\boldsymbol x_0)}\right] \right| \\
&= (n-2) \max\left\{ \left| \theta_2 + \frac{\theta_3}{3}+\frac{\theta_1}{n-2} \right|, \frac{n}{4(n-1)} \left| \theta_2 + \frac{8\theta_1}{n-2} \right| \right\};
\end{align*}
a similar expression also holds for an odd node number \(n>2\).
Consequently, for all fixed parameters excluding
\(\theta_2=\theta_3=0\),
\(\lim_{N\to \infty}\REP_N(\boldsymbol \theta_N)/N=\infty\) then follows
and the graph model with 2-stars and triangles is S-unstable, as
suggested by the breach of Proposition \ref{prp:prop1} for this model
when \(|\theta_2|+|\theta_3|\neq 0\). That is, instability holds even
under \(\theta_2 = - \theta_3/3\) case potentially allowed by
Schweinberger's
(\protect\hyperlink{ref-schweinberger2011instability}{2011}) results.

\subsection{Latent Variable Models of Increasing Parameter
Dimension}\label{latent-variable-models-of-increasing-parameter-dimension}

We next consider instability of discrete data models based on
exponential formulations involving hidden, or latent, variables, such as
those probabilistic graphical models described in Sections
\ref{rbm}-\ref{deep-learning}. We will focus on restricted Boltzmann
machine (RBM) models (Section \ref{rbm}, having one layer of latent
variables for simplicity, though the same instability concepts may be
extended to other deep learning models (Section \ref{deep-learning}. For
\(N\) visible variables
\(\boldsymbol X \equiv \boldsymbol X_N = (X_1,\ldots,X_n)\) as data,
each observation \(X_i\in\{\pm 1\}\) being binary, the RBM-based model
\eqref{eq:RBM2} for \(\boldsymbol X\) is again of FOES-type, though not an
exponential model. However, the distribution of visible variables is
induced by an underlying joint exponential model \eqref{eq:RBM1} for both
visible and latent variables \((\boldsymbol X, \boldsymbol H)\), where
\(\boldsymbol H = (H_1,\ldots,H_{N_{\mathcal{H}}})\) denotes a vector of
\(N_{\mathcal{H}}\) hidden variables (similarly binary). The joint model
is of linear exponential form involving
\(q(N)\equiv N + N_{\mathcal{H}} + N*N_{\mathcal{H}}\) sufficient
statistics given by
\((\boldsymbol X, \boldsymbol H, \boldsymbol X^T\boldsymbol H)\) and
parameters
\(\boldsymbol \theta_N = (\boldsymbol \theta_N^{\mathcal{V}},\boldsymbol \theta_N^{\mathcal{H}}, \boldsymbol \theta_N^{\mathcal{VH}} ) \in\mathbb{R}^{q(N)}\)
corresponding to the \(N\) visible variables \(\boldsymbol X\) (i.e.,
\(\boldsymbol \theta_N^{\mathcal{V}}\in\mathbb{R}^N\)), the
\(N_{\mathcal{H}}\) hidden variables \(\boldsymbol H\) (i.e.,
\(\boldsymbol \theta_N^{\mathcal{H}}\in\mathbb{R}^{N_{\mathcal{H}}}\)),
and the \(N *N_{\mathcal{H}}\) cross-product variables
\(\boldsymbol X^T\boldsymbol H\) (i.e.,
\(\boldsymbol \theta_N^{\mathcal{VH}}\in\mathbb{R}^{N*N_{\mathcal{H}}}\)).
However, unlike some previous exponential models considered in Sections
\ref{one-param-exp}-\ref{fixed-dim-exp} (cf.~Proposition
\ref{prp:prop1}, note that the RBM formulation always associates
parameters with \emph{bounded} statistics (i.e., the components of
\((\boldsymbol X, \boldsymbol H, \boldsymbol X^T\boldsymbol H)\)) so
that model instability cannot arise here due to the magnitude of
sufficient statistics exceeding the sample size \(N\). Instead, RBM
instability may be linked solely to parameter configuration and the fact
that the number \(q(N) \geq N\) of parameters necessarily increases with
the number \(N\) of observations \(\boldsymbol X\), in contrast to
previous exponential cases of fixed parameter dimension.

To highlight the instability issues for the RBM model, consider a simple
model for \(N\) visibles \(\boldsymbol X\) with no hidden variables
(\(N_{\mathcal{H}}=0\)), for which model statements
\eqref{eq:RBM1}-\eqref{eq:RBM2} coincide. An independence model then results
for variables in \(\boldsymbol X\), which has \(q(N)=N\) parameters
\(\boldsymbol \theta_N^{\mathcal{V}} = (\theta_{1,N}^{\mathcal{V}}, \ldots, \theta_{N,N}^{\mathcal{V}}) \in \mathbb{R}^{N}\),
and the measure of model instability becomes \[
\frac{1}{N}\REP(\boldsymbol \theta_N) = \frac{2}{N} \sum_{i=1}^N|\theta_{i,N}^{\mathcal{V}}| \quad \text{(RBM model, no hiddens)}.
\] Hence, this model sequence for \(\boldsymbol X\) will be S-unstable
model if the aggregation of absolute parameters grows faster than the
number \(N\) of parameters/visible variables. Consequently, even for a
simplest RBM model involving independence, preventing instability
requires careful choice of parameters, particularly with regard to how a
parameter configuration differs from zero. For more general RBM models,
the number \(N_{\mathcal{H}}\) of hidden variables \(\boldsymbol H\) can
also be chosen arbitrarily (i.e., as some function
\(N_{\mathcal{H}}\equiv N_{N,\mathcal{H}}\) of \(N\)), which can
substantially inflate the number \(q(N)\) of model parameters and
further impact model instability through accumulated parameters. To
better understand the effects of instability in the RBM structure,
Proposition \ref{prp:prop2} next frames the general behavior of extreme
probabilities in the joint RBM model \eqref{eq:RBM1} for
\((\boldsymbol X, \boldsymbol H)\) and the implied RBM data model
\eqref{eq:RBM2} for \(\boldsymbol X\) alone. Specifically, critical
measures of instability may be closely connected in both models through
tight bounds on their respective LREP values \eqref{eq:elpr}. As a result,
Proposition \ref{prp:prop2} shows how an unstable distribution for
observations \(\boldsymbol X\) may be traced to sources of instability
in the original joint distribution for
\((\boldsymbol X,\boldsymbol H)\). This also suggests a device for
avoiding instability, as provided next.

To state the result, let
\(\REP_{\boldsymbol X}(\boldsymbol \theta_N) \equiv \REP(\boldsymbol \theta_N)\)
denote the LREP value \eqref{eq:elpr} from the marginal distribution
\(P_{\boldsymbol \theta_N}\) of visibles \(\boldsymbol X\) in
\eqref{eq:RBM2} and write the LREP for the joint distribution
\(\tilde{P}_{\boldsymbol \theta_N}\) of
\((\boldsymbol X, \boldsymbol H)\) from \eqref{eq:RBM1} as
\begin{align*}
\REP_{(\boldsymbol X, \boldsymbol H)}(\boldsymbol \theta_N) &=\log\left[  \frac{\max_{ (\boldsymbol x, \boldsymbol h) \in \{\pm 1\}^{N+N_{\mathcal{H}}}}
\tilde{P}_{\boldsymbol \theta_N} (\boldsymbol x, \boldsymbol h)}{\min_{ (\boldsymbol x, \boldsymbol h) \in \{\pm 1\}^{N+N_{\mathcal{H}}}`}
\tilde{P}_{\boldsymbol \theta_N} (\boldsymbol x, \boldsymbol h)} \right] \qquad \text{(joint RBM model)}    \\    
&= \left(\max_{ (\boldsymbol x, \boldsymbol h) \in \{\pm 1\}^{N+N_{\mathcal{H}}}} f_{\boldsymbol \theta_N} (\boldsymbol x, \boldsymbol h)-\min_{ (\boldsymbol x, \boldsymbol h) \in \{\pm 1\}^{N+N_{\mathcal{H}}}} f_{\boldsymbol \theta_N} (\boldsymbol x, \boldsymbol h)\right),
\end{align*}
written as a function
\begin{equation}
\label{eq:f}
f_{\boldsymbol \theta_N} (\boldsymbol x, \boldsymbol h)  \equiv  \sum_{i=1}^N x_i  \theta_{i,N}^{\mathcal{V}} +  \sum_{j=1}^{N_\mathcal{H}} h_j \theta_{j,N}^{\mathcal{H}} + \sum_{i=1}^N \sum_{j=1}^{N_\mathcal{H}} x_i h_j  \theta_{ij,N}^{\mathcal{VH}}  
\end{equation}
of outcomes \(\boldsymbol x =(x_1,\ldots,x_N) \in\{\pm1\}^{N}\),
\(\boldsymbol h =(h_1,\ldots,h_{N_{\mathcal{H}}}) \in\{\pm1\}^{N_{\mathcal{H}}}\)
and parameters
\(\boldsymbol \theta_N \equiv (\boldsymbol \theta_N^{\mathcal{V}},\boldsymbol \theta_N^{\mathcal{H}}, \boldsymbol \theta_N^{\mathcal{VH}})\),
with \(\theta_{i,N}^{\mathcal{V}}\), \(\theta_{j,N}^{\mathcal{H}}\) and
\(\theta_{ij,N}^{\mathcal{VH}}\) denoting respective parameter
components, \(1 \leq i \leq N\), \(1 \leq j \leq N_{\mathcal{H}}\). Due
to the marginalization steps in defining the distribution \eqref{eq:RBM2}
of \(\boldsymbol X\), note that
\(\REP_{\boldsymbol X}(\boldsymbol \theta_N)\) has no immediate
analytical expression similar to that of
\(\REP_{(\boldsymbol X, \boldsymbol H)}(\boldsymbol \theta_N)\). For
clarity, recall also that S-instability \eqref{eq:Sun} in each model type
refers to a respective divergence (i.e.,
\(\lim_{N\to \infty} \REP_{(\boldsymbol X, \boldsymbol H)}(\boldsymbol \theta_N) /(N+N_{\mathcal{H}})=\infty\),
\(\lim_{N\to \infty} \REP(\boldsymbol \theta_N) /N=\infty\)) upon
scaling by the corresponding number of variables in a distribution. In
the following, let \(|\boldsymbol y|_1 = \sum_{i=1}^d |y_i|\) denote the
L-1 norm of a generic vector \(\boldsymbol y =(y_1,\ldots,y_d)\),
\(d \geq 1\). \BeginKnitrBlock{proposition}

\protect\hypertarget{prp:prop2}{}{\label{prp:prop2} }Let
\(P_{\boldsymbol \theta_N}\) denote a RBM-based data model \eqref{eq:RBM2}
for \(N\geq 1\) visible variables
\(\boldsymbol X \equiv \boldsymbol X_N\) derived from
\(\tilde{P}_{\boldsymbol \theta_N}\) as the joint RBM distribution
\eqref{eq:RBM1} of \((\boldsymbol X, \boldsymbol H)\) involving some
number \(N_{\mathcal{H}} \equiv N_{N,\mathcal{H}}\geq 0\) of hidden
variables \(\boldsymbol H \equiv \boldsymbol H_N\) and parameters
\(\boldsymbol \theta_N \equiv (\boldsymbol \theta_N^{\mathcal{V}},\boldsymbol \theta_N^{\mathcal{H}}, \boldsymbol \theta_N^{\mathcal{VH}}) \in\mathbb{R}^{N}\times \mathbb{R}^{N_{\mathcal{H}}} \times \mathbb{R}^{N*N_{\mathcal{H}}}\).
Then,
\begin{enumerate}[(i)]
\item the instability measure $\REP(\boldsymbol \theta_N)$ for the marginal model $P_{\boldsymbol \theta_N}$ of  $\boldsymbol X$ satisfies
    $$
    \left| \REP(\boldsymbol \theta_N)  - \elt\right| \leq    N_{\mathcal{H}}  \log 2
    $$
    for
    $$
    \elt  \equiv   \max_{ \boldsymbol x} \max_{ \boldsymbol h  } f_{\boldsymbol \theta_N} (\boldsymbol x, \boldsymbol h)-\min_{ \boldsymbol x } \max_{ \boldsymbol h }f_{\boldsymbol \theta_N} (\boldsymbol x, \boldsymbol h)
    $$
    based on $f_{\boldsymbol \theta_N}$ from (\ref{eq:f}) with components $\boldsymbol x \in \{\pm 1\}^{N}, \boldsymbol h \in \{\pm 1\}^{N_{\mathcal{H}}}$.
\item The instability measure $\REP_{(\boldsymbol X, \boldsymbol H)}(\boldsymbol \theta_N)\equiv \left(\max_{ \boldsymbol x} \max_{ \boldsymbol h  } f_{\boldsymbol \theta_N} (\boldsymbol x, \boldsymbol h)-\min_{ \boldsymbol x } \min_{ \boldsymbol h }f_{\boldsymbol \theta_N} (\boldsymbol x, \boldsymbol h)\right)$ for the joint model $\tilde{P}_{\boldsymbol \theta_N}$ of $(\boldsymbol X, \boldsymbol H)$ satisfies
    \begin{align*}
    2\Gam +  2|\boldsymbol \theta_N^{\mathcal{H}} |_{1} \; \geq \; \REP_{(\boldsymbol X, \boldsymbol H)}(\boldsymbol \theta_N)    & \geq 
    2\max\big\{\Gam,   |\boldsymbol \theta_N^{\mathcal{H}} |_{1}\big\} \\
    &\geq  2\Gam\\
    &\geq  \elt\\
    &\geq   \max\big\{  \Gamc, \, \Gam  - 2|\boldsymbol \theta_N^{\mathcal{H}} |_{1}  \big\}
    \end{align*}
    for
    $$
    \Gam \equiv \max_{ \boldsymbol h} k_{\boldsymbol \theta_N} (\boldsymbol h)
    \geq  |\boldsymbol \theta_N^{\mathcal{V}} |_{1},\qquad  k_{\boldsymbol \theta_N} (\boldsymbol h)  \equiv \sum_{i=1}^{N }\left| \theta_{i,N}^{\mathcal{V}}   + \sum_{j=1}^{N_{\mathcal{H}}} h_j \theta_{ij,N}^{\mathcal{VH}} \right|,
    $$ 
    and $\Gamc \equiv \min_{ \boldsymbol h} k_{\boldsymbol \theta_N} (\boldsymbol h)$ based on a function  $k_{\boldsymbol \theta_N} (\boldsymbol h)$ of hidden variable outcomes $\boldsymbol h = (h_1,\ldots,h_{N_{\mathcal{H}}})$ and visible-related parameters $\boldsymbol \theta_N^{\mathcal{V}}$ and $\boldsymbol \theta_N^{\mathcal{VH}}$.
\item Assuming $\sup_{N \geq 1} N_{\mathcal{H}}/N<\infty$ additionally, then the following properties 1.-7. hold:
\begin{enumerate}[1.] \itemsep 0cm
    \item an S-unstable visible model $P_{\boldsymbol \theta_N}$ is equivalent to the condition $\lim_{N\to \infty}  \elt/N  =\infty$; further, $P_{\boldsymbol \theta_N}$ is stable when $\elt/N$, $N \geq 1$, is bounded.
    \item an S-unstable joint model $P_{\boldsymbol \theta_N}$ is equivalent to the condition $\lim_{N\to \infty} \max\{|\boldsymbol \theta_N^{\mathcal{H}}|_1, \Gam\}/N =\infty$; further,  $\tilde{P}_{\boldsymbol \theta_N}$ is stable when $[|\boldsymbol \theta_N^{\mathcal{H}}|_1+ \Gam]/N$, $N \geq 1$, is bounded.
    \item if the visible model $P_{\boldsymbol \theta_N}$   is S-unstable, then  the joint model $\tilde{P}_{\boldsymbol \theta_N}$ is also S-unstable.
    \item when $\lim_{N\to \infty}  (|\boldsymbol \theta_N^{\mathcal{V}}|_1-2|\boldsymbol \theta_N^{\mathcal{H}}|_1)/N =\infty$, both $P_{\boldsymbol \theta_N}$ and $\tilde{P}_{\boldsymbol \theta_N}$ are necessarily S-unstable.
    \item when $\lim_{N\to \infty}|\boldsymbol \theta_N^{\mathcal{H}}|_1/N =\infty$,  the joint model $\tilde{P}_{\boldsymbol \theta_N}$ is necessarily S-unstable.
    \item when $\sup_{N \geq 1} |\boldsymbol \theta_N^{\mathcal{H}}|_1 /N<\infty$, the visible model $P_{\boldsymbol \theta_N}$ being S-stable or S-unstable is equivalent to the joint model  $\tilde{P}_{\boldsymbol \theta_N}$ being stable or unstable.
    \item  an S-stable visible model $P_{\boldsymbol \theta_N}$ results if
        $$
        |\boldsymbol \theta_N^{\mathcal{V}}|_1+ |\boldsymbol \theta_N^{\mathcal{VH}} |_1  \leq CN,\quad N \geq 1,
        $$
        for some $C>0$, while an S-stable joint model $\tilde{P}_{\boldsymbol \theta_N}$  results if
        $$
        |\boldsymbol \theta_N|_1 \equiv  |\boldsymbol \theta_N^{\mathcal{V}}|_1+|\boldsymbol \theta_N^{\mathcal{H}}|_1 +|\boldsymbol \theta_N^{\mathcal{VH}} |_1  \leq  C N ,\quad N \geq 1.
        $$
\end{enumerate}
\end{enumerate}
\EndKnitrBlock{proposition}

\BeginKnitrBlock{remark}
\iffalse <span class="remark"><em>Remark. \fi{} The condition
\(\sup_{N \geq 1} N_{\mathcal{H}}/N<\infty\) in Proposition
\ref{prp:prop2}(iii) is often mild in practice (i.e., the number
\(N_{\mathcal{H}}\) of hidden variables is typically not excessively
larger than the number \(N\) of visible observations). This allows
instability results for both marginal and joint RBM models to be more
readily stated together, as the numbers \(N\) and \(N+N_{\mathcal{H}}\)
of variables in these models become asymptotically equivalent.
\EndKnitrBlock{remark}

In Proposition \ref{prp:prop2}(iii), the relationships between RBM
models with regard to instability, and the effects of different
parameter types, follow from the bounds on model instability measures in
Proposition \ref{prp:prop2}(i)-(ii). Generally speaking, all instability
in the marginal RBM model for the data \(\boldsymbol X\) can be
attributed to an excessively large model quantity \(\elt\), which
predominantly follows when main \(\boldsymbol \theta_N^{\mathcal{V}}\)
and interaction \(\boldsymbol \theta_N^{\mathcal{VH}}\) parameters
related to visible variables are too large in magnitude (e.g., upon
accumulation in terms such as
\(|\boldsymbol \theta_N^{\mathcal{V}}|_1\), \(\Gam\) or \(\Gamc\)). For
example, for any bounded sequence
\(|\boldsymbol \theta_N^{\mathcal{H}}|/N\) of hidden parameters, if main
visible parameters \(\boldsymbol \theta_N^{\mathcal{V}}\) are too
extreme (\(|\boldsymbol \theta_N^{\mathcal{V}}|_1/N\to \infty\)), this
aspect will guarantee instability in the visible model
(\(\elt/N\to \infty\)). In fact, the instability measure
\(\elt \equiv \max_{ \boldsymbol x} \max_{ \boldsymbol h } f_{\boldsymbol \theta_N} (\boldsymbol x, \boldsymbol h)-\min_{ \boldsymbol x } \max_{ \boldsymbol h }f_{\boldsymbol \theta_N} (\boldsymbol x, \boldsymbol h)\)
for marginal/visible model represents a clearly smaller portion of the
instability measure
\(\REP_{(\boldsymbol X, \boldsymbol H)}(\boldsymbol \theta_N)\equiv \max_{ \boldsymbol x} \max_{ \boldsymbol h } f_{\boldsymbol \theta_N} (\boldsymbol x, \boldsymbol h)-\min_{ \boldsymbol x } \min_{ \boldsymbol h } f_{\boldsymbol \theta_N} (\boldsymbol x, \boldsymbol h)\)
in the joint RBM model, implying that an unstable marginal model (i.e.,
due to \(\boldsymbol \theta_N^{\mathcal{V}}\),
\(\boldsymbol \theta_N^{\mathcal{VH}}\)) must always translate to an
unstable joint model and that further potential causes of instability
exist for the joint model, often due to the size
\(|\boldsymbol \theta_N^{\mathcal{H}}|_1\). For example, while the joint
RBM model for \((\boldsymbol X,\boldsymbol H)\) must always be unstable
due to a diverging combination of visible and/or interaction parameters
(\(|\boldsymbol \theta_N^{\mathcal{V}}|_1/N\to \infty\) or
\(\Gam/N\to \infty\)) (Proposition \ref{prp:prop2}(iii.2)), instability
for the joint model can also result when the main hidden parameters
\(\boldsymbol \theta_N^{\mathcal{H}}\) become too large relative to
sample size (\(|\boldsymbol \theta_N^{\mathcal{H}}|_1/N\to \infty\) in
Proposition \ref{prp:prop2}(iii.5)). However, under Proposition
\ref{prp:prop2}, the main hidden parameters
\(\boldsymbol \theta_N^{\mathcal{H}}\) do not necessarily entail a
source of instability for the marginal visible model. To explain this
distinction, consider a joint model where all parameters related to
visibles are zero,
\(\boldsymbol \theta_N^{\mathcal{V}}= \boldsymbol \theta_N^{\mathcal{VH}}=\boldsymbol 0\),
but the hidden-related parameters diverge in sum
\(|\boldsymbol \theta_N^{\mathcal{H}}|_1/N\to \infty\). Here, the
explosive behavior among parameters
\(\boldsymbol \theta_N^{\mathcal{H}}\) induces instability in the joint
model for \((\boldsymbol X, \boldsymbol H)\) but the marginal model for
\(\boldsymbol X\), however, has a perfectly stable (and in fact uniform)
distribution in this case. When the hidden parameters are bounded
relative to the sample size
(\(\sup_{N\geq 1} |\boldsymbol \theta_N^{\mathcal{H}}|_1/N<\infty\)),
then all instability in both the joint and marginal RBM models can be
directly linked to excessively large visible
\(\boldsymbol \theta_N^{\mathcal{V}}\) and/or interaction parameters
\(\boldsymbol \theta_N^{\mathcal{VH}}\) so that features of
stability/instability must be the same across both models (Proposition
\ref{prp:prop2}(iii.6)). Hence, to prevent instability in the joint
model, the combined magnitudes of all parameters \(\theta_N\) must be
controlled (cf.~Proposition \ref{prp:prop2}(iii.7)), while a stable
visible data model technically results in constraining only the sizes of
visible-related parameters \(\boldsymbol \theta_N^{\mathcal{V}}\),
\(\boldsymbol \theta_N^{\mathcal{VH}}\). Nevertheless, because the joint
model is often employed in practice for purposes of simulation and
simulation-based inference, it is still reasonable to consider parameter
choices for ensuring a stable joint model (and, consequently, a stable
visible model as well). Further evidence of this is seen in the
following numerical example. \par
\begin{figure}
\includegraphics{note_files/figure-latex/rbm-plots-1} \includegraphics{note_files/figure-latex/rbm-plots-2} \caption{The sample mean value of $\text{ELPR}(\boldsymbol \theta)/N_{\mathcal{V}}$ (left) and $\DN(\boldsymbol \theta)$ at each grid point for each combination of magnitude of $\boldsymbol \theta$. As the magnitude of $\boldsymbol \theta$ grows, so does the value of these metrics, indicating typical instability in the model.}\label{fig:rbm-plots}
\end{figure}
In our numerical experiment, we allow the two types of terms (main
effects terms corresponding to visible and hidden parameters
\(\boldsymbol \theta_{main} = (\boldsymbol \theta_N^{\mathcal{V}}, \boldsymbol \theta_N^{\mathcal{H}})\)
and interaction parameters \(\boldsymbol \theta_N^{\mathcal{VH}}\)) to
have varying average magnitudes,
\(||\boldsymbol \theta_{main} || /(N_{\mathcal{H}}+N_{\mathcal{V}})\)
and
\(||\boldsymbol \theta_{interaction} || /(N_{\mathcal{H}}*N_{\mathcal{V}})\)
for a RBM with \(N_\mathcal{V} = 9\) visibles and \(N_\mathcal{H} = 5\)
hiddens. These average magnitudes vary on a grid between 0.001 and 3
with 20 breaks, yielding 400 grid points. At each point in the grid, 100
vectors (\(\boldsymbol \theta_{main}\)) are sampled uniformly on a
sphere with radius corresponding to the first coordinate in the grid and
100 vectors (\(\boldsymbol \theta_{interaction}\)) are sampled uniformly
on a sphere with radius corresponding to the second coordinate in the
grid via sums of squared and scaled iid Normal\((0, 1)\) variables.
These vectors are then paired to create 100 values of
\(\boldsymbol \theta_N\) with magnitudes at each point in the grid. The
values \(\REP(\boldsymbol \theta_N)/N_{\mathcal{V}}\) and
\(\DN(\boldsymbol \theta_N)\) are then calculated for each
\(\boldsymbol \theta_N\) and then summarized for each point in the grid
using the sample mean. The results of this numerical study are shown in
Figure \ref{fig:rbm-plots}. From these two plots, it is clear that for
larger magnitudes of the parameter vectors, there is evidence of
S-instability in that the log-ratio of extremal probabilities scaled by
\(N_{\mathcal{V}}\) and the the biggest log-probability ratio for a one
component change in data outcomes are both increasing away from
\(\boldsymbol \theta_N = \boldsymbol 0\), further supporting
\ref{prp:prop2}(iii.2 and iii.5).

In more complicated graphical models involving further or deeper hidden
layers, the same issues and causes of instability similarly exist, but
are compounded by a greater number of model parameters. S-unstable joint
models will similarly follow if the combined sizes of all parameters are
too great relative to the total number of variables, while instability
in the data model for visible variables will depend only on the main or
interaction parameters directly related to visibles and how their
accumulated magnitude compares to the observation sample size \(N\).

\section{Statistical Consequences of Instability}\label{implications}

Due to their extreme sensitivity in probability structure, S-instability
in FOES models may often translate to numerical complications, and in
fact obstructions, in both simulation and statistical inference based on
likelihoods. We describe these aspects in Sections
\ref{mcmc}-\ref{bayes} with regard to data simulation, maximum
likelihood estimation and Bayes inference, respectively.

\subsection{Implications for Simulation}\label{mcmc}

Suppose one aims to apply MCMC to simulate data
\(\boldsymbol X=(X_1,\ldots,X_n)\) from a FOES model
\(P_{\boldsymbol \theta_N}(\boldsymbol x)\),
\(\boldsymbol x \in\mathcal{X}^N\) whereby a chain is constructed with
\(P_{\boldsymbol \theta_N}\) as the stationary distribution. For an
unstable FOES model, one component changes in outcomes may produce
radically different probabilities, which may then impose numerical
barriers to MCMC. For example, consider implementation of a Gibbs
sampler from the full conditional distributions
\(P_{\boldsymbol \theta_N} (X_i = x| \boldsymbol x_{-i})\),
\(x\in\mathcal{X}\), of each variable \(X_i\) based on values
\(\boldsymbol x_{-i}\in \mathcal{X}^{N-1}\) for the remaining variables,
say \(\boldsymbol X_{-i}\), in \(\boldsymbol X\). If a single change in
\(X_i\) from one value \(x_1\) to another \(x_2\) may produce two
outcomes \(\boldsymbol x^{(1)}\) and \(\boldsymbol x^{(2)}\) for
\(\boldsymbol X\) with vastly different probabilities under the joint
distribution \(P_{\boldsymbol \theta_N}\), then the Gibbs sampler can
have extreme log-ratios in its transition probabilities, \[
\log \left| \frac{P_{\boldsymbol \theta_N} (X_i =x_1 | \boldsymbol x^{(1)}_{-i})}{P_{\boldsymbol \theta_N} (X_i = x_2| \boldsymbol x^{(2)}_{-i}) } \right|=\log \left| \frac{P_{\boldsymbol \theta_N} ( \boldsymbol x^{(1)} )}{P_{\boldsymbol \theta_N} (  \boldsymbol x^{(2)} ) } \right|,
\]\\
as conditional probabilities are proportional to joint probabilities
that, with unstable models, can have unbounded log-probability ratios in
one-component changes (Theorem \ref{thm:instab-elpr}). This can hinder
the ability of chain to effectively explore the sample space of the
observations \(\boldsymbol X\), as the chain may mix poorly by moving
rapidly to, and slowly away from, sections of the sample space. In this
case, for example, the Markov chain may become entrapped within a mode
of the probability function, with rare chance of escaping to adequately
mimic the occupation frequencies in the overall sample space. If modes
of the unstable model are not unique, then important outcomes may be
missed without multiple chains or impractically enormous numbers of MCMC
samples. This mixing problem is due to the unstable stationary
distribution (unbounded ratios of probabilities under the joint model),
rather than in any particulars of the MCMC algorithm, and similar
complications can also arise for Metropolis-Hastings algorithms for
MCMC. Hence, while an unstable FOES model \(P_{\boldsymbol \theta_N}\)
may be valid and technically open to simulation by MCMC, the aspect of
instability can render applications of MCMC as numerically infeasible
for simulation purposes. This result is in line with conclusions of
Handcock (\protect\hyperlink{ref-handcock2003assessing}{2003}) and
Schweinberger
(\protect\hyperlink{ref-schweinberger2011instability}{2011}) for other
exponential models.

\subsection{Implications for Maximum Likelihood
Inference}\label{implications-for-maximum-likelihood-inference}

Volatility in the probability structure of an unstable model can also
hamper efforts to maximize likelihood functions in statistical
inference. When a FOES model is unstable along a parameter sequence
\(\boldsymbol \theta_N\), the same model can further be unstable along
parameters \(-\boldsymbol \theta_N\) in an opposite direction from the
origin (model condition PSR and Corollary \ref{cor:sign}). This can
translate into potential sensitivity of the likelihood function around
zero, and lead to numerical complications in maximizing the objective
function. We next provide a discussion of this issue in a way that
builds upon and extends related findings by Schweinberger
(\protect\hyperlink{ref-schweinberger2011instability}{2011}), who
largely focused on the case of one-parameter exponential models.

With many probability models, the modes and anti-modes in the
probability structure under one parameter \(\boldsymbol \theta_N\) are
reversed in role when the parameter sign changes
\(-\boldsymbol \theta_N\). Because unstable models tend to degeneracy,
the opposite signed parameters further push unstable models to assign
nearly all probability to extremely opposite data configurations, given
by modes/anti-modes. This is made concrete in Theorem
\ref{thm:something}, relating the degeneracy from unstable models to the
expected behavior of log-likelihood functions.

\BeginKnitrBlock{theorem}
\protect\hypertarget{thm:something}{}{\label{thm:something} }Let
\(P_{\boldsymbol \theta_N}\), \(N \geq 1\), denote an S-unstable FOES
model sequence, which additionally satisfies model condition PSR. Let
\(\boldsymbol{x}_{\max, \boldsymbol \theta_N},\boldsymbol{x}_{\min, \boldsymbol \theta_N}\in\mathcal{X}^N\)
denote, respectively, a mode and anti-mode of the model
\(P_{\boldsymbol \theta_N}(\boldsymbol x)\),
\(\boldsymbol x\in\mathcal{X}^N\), for \(N\) observations
\(\boldsymbol X = (X_1,\ldots,X_N)\), whereby
\(P_{\boldsymbol \theta_N }(\boldsymbol{x}_{\max, \boldsymbol \theta_N}) = \max_{\boldsymbol y\in\mathcal{X}^N} P_{\boldsymbol \theta_N }(\boldsymbol y)\)
and
\(P_{\boldsymbol \theta_N }(\boldsymbol{x}_{\min, \boldsymbol \theta_N}) = \min_{\boldsymbol y\in\mathcal{X}^N} P_{\boldsymbol \theta_N }(\boldsymbol y)\).

Then, letting \(\stackrel{p, \mathrm{E}}{\longrightarrow}\) denote
convergence in probability and expectation, as \(N\to \infty\), \[
\frac{1}{\REP(\boldsymbol \theta_N)} \log\left[ \frac{P_{\boldsymbol \theta_N }(\boldsymbol X) }{ \displaystyle{\min_{\boldsymbol y\in\mathcal{X}^N}} P_{\boldsymbol \theta_N }(\boldsymbol y)}\right] = \frac{ \log P_{\boldsymbol \theta_N } ( \boldsymbol X)  - \log P_{\boldsymbol \theta_N }(\boldsymbol{x}_{\min, \boldsymbol \theta_N}) }
{ \log P_{\boldsymbol \theta_N } (\boldsymbol{x}_{\max, \boldsymbol \theta_N})  - \log P_{\boldsymbol \theta_N }(\boldsymbol{x}_{\min, \boldsymbol \theta_N}) }
\stackrel{p, \mathrm{E}}{\longrightarrow}
\; 1
\] under \(\boldsymbol\theta_N\) while \[
\frac{1}{\REP(-\boldsymbol \theta_N)} \log\left[ \frac{P_{-\boldsymbol \theta_N }(\boldsymbol X) }{  \displaystyle{\min_{\boldsymbol y\in\mathcal{X}^N}} P_{-\boldsymbol \theta_N }(\boldsymbol y)} \right]
=    \frac{ \log P_{-\boldsymbol \theta_N } ( \boldsymbol X) - \log P_{ \boldsymbol \theta_N }(\boldsymbol{x}_{\max, \boldsymbol \theta_N})    }
{ \log P_{-\boldsymbol \theta_N } (\boldsymbol{x}_{\min, \boldsymbol \theta_N})  - \log P_{-\boldsymbol \theta_N }(\boldsymbol{x}_{\max, \boldsymbol \theta_N}) }
\stackrel{p,\mathrm{E}}{\longrightarrow}
\; 1,
\] under \(-\boldsymbol\theta_N\), where \[
\REP( \boldsymbol \theta_N) \equiv \log \frac{P_{\boldsymbol \theta_N } (\boldsymbol{x}_{\max, \boldsymbol \theta_N})}{P_{\boldsymbol \theta_N }(\boldsymbol{x}_{\min, \boldsymbol \theta_N})} = \log  \frac{P_{-\boldsymbol \theta_N } (\boldsymbol{x}_{\min, \boldsymbol \theta_N}) }{ P_{-\boldsymbol \theta_N }(\boldsymbol{x}_{\max, \boldsymbol \theta_N})}=   \REP( -\boldsymbol \theta_N), \quad N \geq 1.
\]
\EndKnitrBlock{theorem}

Theorem \ref{thm:something} entails log-likelihood functions based on
unstable models are both inversely related and degenerate at opposited
signed parameters \(\boldsymbol \theta_N\) or \(-\boldsymbol \theta_N\),
so that likelihoods are highest at different extremes in data
configuration (e.g., \(\boldsymbol{x}_{\max, \boldsymbol \theta_N}\)
under \(\boldsymbol \theta\)-probabilities or
\(\boldsymbol{x}_{\min, \boldsymbol \theta_N}\) under
\(-\boldsymbol \theta\)-probabilities). If the observed outcome
\(\boldsymbol x\) for data \(\boldsymbol X\) is not a mode/anti-mode,
then probabilities for the outcome may be small under both parameters
\(\boldsymbol \theta_N\) and \(-\boldsymbol \theta_N\), in which case
associated optimization steps may then shift around zero and struggle to
converge. In many model formulations, the zero parameter
\(\boldsymbol \theta_N=\boldsymbol 0\) is a ``safe'' position among
parameters, representing a guaranteed stable model (having uniform
probabilities among outcomes), which can also tether a broad parameter
search attempted among unstable models. Handcock
(\protect\hyperlink{ref-handcock2003assessing}{2003}) describes similar
results for degenerate exponential models, and Theorem
\ref{thm:something} also supports an important finding of Schweinberger
(\protect\hyperlink{ref-schweinberger2011instability}{2011} Corollary 1)
for one-parameter linear exponential models \eqref{eq:expo}. In the latter
case, the likelihood score function at \(\boldsymbol \theta_N\) is the
expected value
\(\mu(\boldsymbol \theta_N)\equiv \mathrm{E}_{\boldsymbol \theta_N} g(\boldsymbol X)\)
of the sufficient statistic \(g(\cdot)\), and optimization involves
solving \(\mu(\cdot)=g(\boldsymbol x)\) for an observed outcome
\(\boldsymbol x\). For unstable models in this exponential class,
Schweinberger
(\protect\hyperlink{ref-schweinberger2011instability}{2011} Corollary 1)
shows that \[
\lim_{N\to \infty}\frac{\mu(\boldsymbol \theta_N) -L_n}{U_n-L_N}= \left\{ \begin{array}{lcl}
1 && \text{for } \boldsymbol \theta_N>0,\\
0 && \text{for } \boldsymbol \theta_N<0, \end{array}\right.
\] where again \(U_N\) and \(L_N\) denote the maximum and minimum values
of the statistic \(g(\boldsymbol x)\),
\(\boldsymbol x\in\mathcal{X}^N\). As described by Schweinberger
(\protect\hyperlink{ref-schweinberger2011instability}{2011}), the
implication for maximum likelihood estimation is that, unless an
observed outcome \(\boldsymbol x\) falls at an extreme \(U_N, L_N\)
(i.e., modes/anti-modes), optimization steps in the parameter space can
iterate in relatively small increments around zero and fail to converge.
For unstable one-parameter exponential models, the maximum likelihood
results of Schweinberger
(\protect\hyperlink{ref-schweinberger2011instability}{2011}) turn out to
be a special case of Theorem \ref{thm:something} and the LREP expansion
\eqref{eq:UL} in this setting; namely, for an unstable model with
\(\boldsymbol \theta_N>0\), \[
\frac{1}{\REP(\boldsymbol \theta_N)} \log\left[ \frac{P_{\boldsymbol \theta_N }(\boldsymbol X) }{\displaystyle{\min_{\boldsymbol y\in\mathcal{X}^N}} P_{\boldsymbol \theta_N }(\boldsymbol y)}\right] = \frac{\boldsymbol g(\boldsymbol X) - L_N}{U_N-L_N}\;\;\stackrel{p, \mathrm{E}}{\longrightarrow}\;\; 1
\] holds as \(N\to \infty\) by Theorem \ref{thm:something}, while under
\(-\boldsymbol \theta_N<0\) \[
\frac{1}{\REP(-\boldsymbol \theta_N)} \log\left[ \frac{P_{-\boldsymbol \theta_N }(\boldsymbol X) }{\displaystyle{\min_{\boldsymbol y\in\mathcal{X}^N}} P_{-\boldsymbol \theta_N }(\boldsymbol y)}\right] = \frac{U_N -\boldsymbol g(\boldsymbol X)}{U_N-L_N} = 1 - \frac{\boldsymbol g(\boldsymbol X) - L_N}{U_N-L_N} \;\;\stackrel{p, \mathrm{E}}{\longrightarrow}\;\; 1.
\] Again, when all probability in unstable models may be pushed to
opposite extremes in the sample space, due to a combination of
degeneracy and parameter sign, numerical complications in likelihood
maximization may occur.

\subsection{Implications for Bayes Inference}\label{bayes}

The potential numerical difficulties with maximum likelihood with
unstable models, as described in the previous section, can naturally
carry over to Bayes inference. Considering that the degeneracy issues
related to unstable models can cause likelihoods can be flat (e.g., near
zero) for many parameters under a given data outcome and that sign
changes in parameters can shift tremendous probability to extreme and
opposite outcomes in the sample space (e.g., Corollary \ref{cor:sign},
Theorem \ref{thm:something}), then numerical complications may arise
with Bayes inference in sampling a posterior parameter space based on
MCMC. The potential challenges in chain mixing are similar to those
presented in Section \ref{mcmc}, though in chain movements through the
parameter space as opposed to the sample space in data generation. That
is, in the Bayes setting for sampling a posterior distribution for
\(\boldsymbol \theta_N\), a chain may unstable to effectively explore
the parameter space due partly to extreme and potentially unbounded
probability ratios from parameter sign changes, which represents a
parameter space analog to how one component changes in the sample space
may impact data simulation with unstable models. For example, if
\(\pi(\cdot)\) denotes a prior density for \(\boldsymbol \theta_N\) and
\(q(\cdot | \cdot)\) denotes a proposal distribution for use in a
Metropolis-Hastings (MH) sampler, then MH acceptance probability becomes
\[
\alpha\left(\boldsymbol \theta^{(1)} \mid \boldsymbol \theta^{(2)}\right)= \min\left\{1,
\frac{q(\boldsymbol \theta_N^{(2)} \mid \boldsymbol \theta_N^{(1)})}{q(\boldsymbol \theta_N^{(1)} \mid \boldsymbol \theta_N^{(2)})}
\frac{P_{\boldsymbol \theta_N^{(1)}} ( \boldsymbol x ) \pi(\boldsymbol \theta_N^{(1)}) }{P_{\boldsymbol \theta^{(2)}_N} (  \boldsymbol x) \pi(\boldsymbol \theta_N^{(2)}) } \right\},
\] which indicates how parameter sensitivity in the likelihood
\(P_{\boldsymbol \theta_N} ( \boldsymbol x)\) may complicate sampling of
the posterior
\(P_{\boldsymbol \theta_N} ( \boldsymbol x) \pi(\boldsymbol \theta_N)\)
(i.e., moving from \(\boldsymbol \theta^{(1)}\) to
\(\boldsymbol \theta^{(2)}\) in the parameter space). Furthermore, the
potential for model instability and the size of the parameter space can
also become greater with the introduction of latent variables to
existing data variables, as involved in some model formulations
described in Sections \ref{rbm}-\ref{deep-learning}. As latent variables
are often sampled with parameters in a Bayes MCMC approach, this aspect
may further compound numerical problems in chain mixing.

\section{Concluding Remarks}\label{conclusions}

For a large class of models that covers a broad range of applications
(including ``deep learning''), we have developed a formal definition of
instability in model probability structure and elucidated multiple
consequences of instability. We have shown for FOES models that
instability manifests through small changes in data leading to
potentially large changes in probability as well as the potential to
place all probability on certain modal subsections of the sample space,
which potentially could be small. Such instability is often due to a
complex interaction between the model statistics used (i.e., how
numerous and large these may become) and the number and magnitudes of
parameters in the model formulation. For many FOES models, the
possibility exists, at least in principle, to constraint parameters in a
way balances their potential contributions against those of model
statistics in order to prevent probability instabilities. The FOES model
class is quite broad and, in developing such models for large data sets,
some caution should be used in parameter specification to control
effects of model instability.

\clearpage

\appendix


\section{Proofs of instability results}\label{appendix-instab}

\textbf{Proof of Proposition \ref{thm:instab-elpr}.} For part (i), we
prove the contrapositive, supposing that
\(\DN(\boldsymbol \theta_N) \le C\) holds for some \(C > 0\) and show
\(\REP(\boldsymbol \theta_N) \leq NC\). Let
\(\boldsymbol x_{min} \equiv \argmin\limits_{\boldsymbol x \in \mathcal{X}^N}P_{\boldsymbol \theta_N}(\boldsymbol x)\)
and
\(\boldsymbol x_{max} \equiv \argmax\limits_{\boldsymbol x \in \mathcal{X}^N}P_{\boldsymbol \theta_N}(\boldsymbol x)\).
Note there exists a sequence
\(\boldsymbol x_{min} \equiv \boldsymbol x_0, \boldsymbol x_1, \dots, \boldsymbol x_k \equiv \boldsymbol x_{max}\)
in \(\mathcal{X}^N\) of component-wise switches to move from
\(\boldsymbol x_{min}\) to \(\boldsymbol x_{max}\) in the sample space
(i.e. \(\boldsymbol x_i, \boldsymbol x_{i + 1} \in \mathcal{X}^N\)
differ in exactly \(1\) component, \(i = 0, \dots, k\)) for some integer
\(k \in \{0, 1, \dots, N\}\). Under the FOES model, recall
\(P_{\boldsymbol \theta_N}(\boldsymbol x) > 0\) holds so that
\(\log P_{\boldsymbol \theta_N}(\boldsymbol x)\) is well-defined for
each outcome \(\boldsymbol x \in \mathcal{X}^N\). Then, if \(k > 0\), it
follows that
\begin{align*}
\REP(\boldsymbol \theta_N) = \log\left[\frac{P_{\boldsymbol \theta_N}(\boldsymbol x_{max})}{P_{\boldsymbol \theta_N}(\boldsymbol x_{min})}\right] &= \left|\sum\limits_{i = 1}^k\log\left(\frac{P_{\boldsymbol \theta_N}(\boldsymbol x_i)}{P_{\boldsymbol \theta_N}(\boldsymbol x_{i-1})}\right)\right| \\
&\le \sum\limits_{i = 1}^k\left|\log\left(\frac{P_{\boldsymbol \theta_N}(\boldsymbol x_i)}{P_{\boldsymbol \theta}(\boldsymbol x_{i-1})}\right)\right| \le k \Delta_N(\boldsymbol \theta_N) \le NC,
\end{align*}
using \(k \le N\) and \(\Delta(\boldsymbol \theta_N) \le C\). If
\(k = 0\), then \(\boldsymbol x_{max} = \boldsymbol x_{min}\) and the
same bound above holds. This establishes part (i). To show part (ii),
note the definition of S-instability (i.e.,
\(\lim_{N\to \infty}\REP(\boldsymbol \theta_N)/N= \infty\)) combined
with part (i) implies that
\(\lim_{N\to \infty}\DN(\boldsymbol \theta_N)=\infty\). \hfill \(\Box\)

\textbf{Proof of Proposition \ref{thm:degenFOES}.} As
\(|\mathcal{X}|<\infty\) holds in the FOES model, we may suppose
\(|\mathcal{X}|>1\); otherwise, \(\mathcal{X}^N\) has one outcome and
the model is trivially degenerate for all \(N \geq 1\). Fix
\(0 < \epsilon < 1\) and write
\(\boldsymbol x_{min} \equiv \argmin\limits_{\boldsymbol x \in \mathcal{X}^N}P_{\boldsymbol \theta_N}(\boldsymbol x)\)
and
\(\boldsymbol x_{max} \equiv \argmax\limits_{\boldsymbol x \in \mathcal{X}^N}P_{\boldsymbol \theta_N}(\boldsymbol x)\).
Then, \(\boldsymbol x_{max} \in M_{\epsilon, \boldsymbol \theta_N}\), so
\(P_{\boldsymbol \theta_N}(M_{\epsilon, \boldsymbol \theta_N}) \ge P_{\boldsymbol \theta_N}(\boldsymbol x_{max}) > 0\).
If
\(\boldsymbol x \in \mathcal{X}^N \setminus M_{\epsilon, \boldsymbol \theta_N}\),
then by definition
\(P_{\boldsymbol \theta_N}(\boldsymbol x) \le [P_{\boldsymbol \theta_N}(\boldsymbol x_{max})]^{1-\epsilon}[P_{\boldsymbol \theta_N}(\boldsymbol x_{min})]^{\epsilon}\)
holds so that \[
1-P_{\boldsymbol \theta_N}(M_{\epsilon, \boldsymbol \theta_N})
 = \sum\limits_{\boldsymbol x \in \mathcal{X}^N \setminus M_{\epsilon, \boldsymbol \theta_N}}P_{\boldsymbol \theta_N}(\boldsymbol x)
  \le (|\mathcal{X}|^N)[P_{\boldsymbol \theta_N}(\boldsymbol x_{max})]^{1-\epsilon}[P_{\boldsymbol \theta_N}(\boldsymbol x_{min})]^{\epsilon}.
\] From the lower bound on
\(P_{\boldsymbol \theta_N}(M_{\epsilon, \boldsymbol \theta_N})\) and the
upper bound on
\(1-P_{\boldsymbol \theta_N}(M_{\epsilon, \boldsymbol \theta_N})\), it
follows that
\begin{align*}
\frac{1}{N}\log\left[\frac{P_{\boldsymbol \theta_N}(M_{\epsilon, \boldsymbol \theta_N})}{1-P_{\boldsymbol \theta_N}(M_{\epsilon, \boldsymbol \theta_N})}\right] & \ge \frac{1}{N} \log\left[\frac{P_{\boldsymbol\theta_N}(\boldsymbol x_{max})}{(|\mathcal{X}|^N)[P_{\boldsymbol \theta_N}(\boldsymbol x_{max})]^{1-\epsilon}[P_{\boldsymbol \theta_N}(\boldsymbol x_{min})]^{\epsilon}}\right] \\
&= \frac{\epsilon}{N} \log\left[\frac{P_{\boldsymbol \theta_N}(\boldsymbol x_{max})}{P_{\boldsymbol \theta_N}(\boldsymbol x_{min})}\right] - \log |\mathcal{X}| \rightarrow \infty
\end{align*}
as \(N \rightarrow \infty\) by the definition of an S-unstable FOES
model \eqref{eq:Sun}. Consequently,
\(P_{\boldsymbol \theta_N}(M_{\epsilon, \boldsymbol \theta_N}) \rightarrow 1\)
as \(N \rightarrow \infty\) as claimed. \hfill \(\Box\)

\textbf{Proof of Corollary \ref{cor:sign}.} The model condition PSR
implies that
\begin{equation}
\label{eq:R}
\frac{\max_{\boldsymbol y \in \mathcal{X}^N}P_{\boldsymbol \theta_N}(\boldsymbol y)  }{\min_{\boldsymbol y \in \mathcal{X}^N}P_{\boldsymbol \theta_N}(\boldsymbol y) } = \frac{\max_{\boldsymbol y \in \mathcal{X}^N}P_{-\boldsymbol \theta_N}(\boldsymbol y) }{\min_{\boldsymbol y \in \mathcal{X}^N}P_{-\boldsymbol \theta_N}(\boldsymbol y) }
\end{equation}
so that the log-ratio
\(\REP(\boldsymbol \theta_N)=\REP(-\boldsymbol \theta_N)\) is the same
for both \(\boldsymbol \theta_N\) and \(-\boldsymbol \theta_N\) in
\eqref{eq:elpr}. Now part (i) of Corollary \ref{cor:sign} follows from
\(\REP(\boldsymbol \theta_N)/N=\REP(-\boldsymbol \theta_N)/N\to \infty\)
as \(N\to \infty\) in \eqref{eq:Sun}. To show part (ii), fix
\(0 < \epsilon < 1\) and consider a \(\epsilon\)-mode set
\(\mathcal{M}_{\epsilon, \boldsymbol \theta_N}\) under
\(\boldsymbol \theta_N\) from \eqref{eq:mode}. If
\(\boldsymbol x \in \mathcal{M}_{\epsilon, \boldsymbol \theta_N}^c \equiv \mathcal{X}^N \setminus \mathcal{M}_{\epsilon, \boldsymbol \theta_N}\),
then, by definition, \[
\frac{P_{\boldsymbol \theta_N}(\boldsymbol x)}{\min_{\boldsymbol y \in \mathcal{X}^N}P_{\boldsymbol \theta_N}(\boldsymbol y)} \leq \left[\frac{\max_{\boldsymbol y \in \mathcal{X}^N}P_{\boldsymbol \theta_N}(\boldsymbol y)  }{\min_{\boldsymbol y \in \mathcal{X}^N}P_{\boldsymbol \theta_N}(\boldsymbol y) } \right]^{1-\epsilon}
\] holds, which is equivalent to \[
\frac{\max_{\boldsymbol y \in \mathcal{X}^N}P_{-\boldsymbol \theta_N}(\boldsymbol y)}{P_{-\boldsymbol \theta_N}(\boldsymbol x)} \leq
\left[\frac{\max_{\boldsymbol y \in \mathcal{X}^N}P_{-\boldsymbol \theta_N}(\boldsymbol y)  }{\min_{\boldsymbol y \in \mathcal{X}^N}P_{-\boldsymbol \theta_N}(\boldsymbol y) } \right]^{1-\epsilon}
\] by model condition PSR and \eqref{eq:R}. The latter is in turn
equivalent to
\begin{equation}
\label{eq:R2}
\log P_{-\boldsymbol \theta_N}(\boldsymbol x) \geq \epsilon \max\limits_{\boldsymbol y \in \mathcal{X}^N} \log  P_{-\boldsymbol \theta_N}(\boldsymbol y) + (1-\epsilon)\min\limits_{\boldsymbol y \in \mathcal{X}^N} \log P_{-\boldsymbol \theta_N}(\boldsymbol y),
\end{equation}
so that
\(\boldsymbol x \in \mathcal{M}_{\epsilon, \boldsymbol \theta_N}^c\) if
and only if \eqref{eq:R2} holds. Next consider the \((1-\epsilon)\)-mode
set \(\mathcal{M}_{1-\epsilon, -\boldsymbol \theta_N}\) under
\(-\boldsymbol \theta_N\) from \eqref{eq:mode}. If
\(\boldsymbol x \in\mathcal{M}_{1-\epsilon, -\boldsymbol \theta_N}\),
then by definition \(\boldsymbol x\) fulfills \eqref{eq:R2} and so
\(\boldsymbol x \in \mathcal{M}_{\epsilon, \boldsymbol \theta_N}^c\),
showing that
\(\mathcal{M}_{1-\epsilon, -\boldsymbol \theta_N} \subset \mathcal{M}_{\epsilon, \boldsymbol \theta_N}^c\).
By this and the fact that that Theorem \ref{thm:degenFOES} and Corollary
\ref{cor:sign}(i) entail that
\(P_{-\boldsymbol \theta_N}(\boldsymbol X_N \in \mathcal{M}_{1-\epsilon, -\boldsymbol \theta_N})\to 1\)
as \(N\to \infty\) (i.e., \(P_{-\boldsymbol \theta_N}\) is S-unstable),
we have \[
1  = \lim_{N\to \infty} P_{-\boldsymbol \theta_N}(\boldsymbol X_N \in \mathcal{M}_{1-\epsilon, -\boldsymbol \theta_N}) \leq \lim_{N\to \infty} P_{-\boldsymbol \theta_N}(\boldsymbol X_N \in \mathcal{M}_{\epsilon, \boldsymbol \theta_N}^c ) \leq 1,
\] proving Corollary \ref{cor:sign}(ii) \hfill \(\Box\)

\textbf{Proof of Proposition \ref{prp:prop1}.} For any two outcomes
\(\boldsymbol x_1, \boldsymbol x_2\in\mathcal{X}^N\), the log-ratio of
probabilities from the linear exponential model \eqref{eq:expo} with \(k\)
parameters/statistics satisfies \[
\left|\log \left[ \frac{P_{\boldsymbol \theta_N}(\boldsymbol x_1)}{P_{\boldsymbol \theta_N}(\boldsymbol x_2)}  \right] \right| =
\left|  \sum_{i=1}^k \theta_{i,N} [g_{i,k}(\boldsymbol x_1) - g_{i,k}(\boldsymbol x_2) ] \right|  \leq  \sum_{i=1}^k | \theta_{i,N}| (U_{i,N}-L_{i,N});
\] consequently,
\(\REP(\boldsymbol \theta_N ) \leq \sum_{i=1}^k | \theta_{i,N}| (U_{i,N}-L_{i,N})\)
holds in \eqref{eq:elpr} and model stability in Proposition
\ref{prp:prop1} follows from \eqref{eq:Sun}. \hfill \(\Box\)

\textbf{Proof of Proposition \ref{prp:prop2}.} Writing
\(\boldsymbol x=(x_1,\ldots,x_N)\) and
\(\boldsymbol h = (h_1,\ldots,h_{N_{\mathcal{H}}})\) with all components
\(x_i,h_j\in\{\pm 1\}\), probabilities in the joint RBM model
\eqref{eq:RBM1} can be written as
\(\tilde{P}_{\boldsymbol \theta_N} (\boldsymbol x, \boldsymbol h) = c(\boldsymbol \theta_N)\exp[ f_{\boldsymbol \theta_N} (\boldsymbol x, \boldsymbol h)]\)
in terms of the function
\(f_{\boldsymbol \theta_N} (\boldsymbol x, \boldsymbol h)\) from
\eqref{eq:f} and the normalizing constant
\(c(\boldsymbol \theta_N)= \exp [-\psi(\boldsymbol \theta_N)]\) from
\eqref{eq:RBM1}. Let \(\boldsymbol x_M, \boldsymbol x_m\in\{\pm 1\}^N\) be
such that
\(P_{\boldsymbol \theta_N} (\boldsymbol x_M) = \max_{\boldsymbol x}P_{\boldsymbol \theta_N} (\boldsymbol x)\)
and
\(P_{\boldsymbol \theta_N} (\boldsymbol x_m) = \min_{\boldsymbol x}P_{\boldsymbol \theta_N} (\boldsymbol x)\)
under the marginal RBM model
\(P_{\boldsymbol \theta_N} (\boldsymbol x) = c(\boldsymbol \theta_N)\sum_{\boldsymbol h \in\{\pm 1\}^{\mathcal{N}_H}} \tilde{P}_{\boldsymbol \theta_N} (\boldsymbol x, \boldsymbol h)= c(\boldsymbol \theta_N)\sum_{\boldsymbol h \in\{\pm 1\}^{\mathcal{N}_H}} \exp[ f_{\boldsymbol \theta_N} (\boldsymbol x, \boldsymbol h)]\)
from \eqref{eq:RBM2}. Also, \(\boldsymbol x_0,x_1\in\{\pm 1\}^N\) be such
that
\(\max_{\boldsymbol h}f_{\boldsymbol \theta_N} (\boldsymbol x_0, \boldsymbol h)=\max_{\boldsymbol x}\max_{\boldsymbol h}f_{\boldsymbol \theta_N} (\boldsymbol x , \boldsymbol h)\)
and
\(\max_{\boldsymbol h}f_{\boldsymbol \theta_N} (\boldsymbol x_1, \boldsymbol h)=\min_{\boldsymbol x}\max_{\boldsymbol h}f_{\boldsymbol \theta_N} (\boldsymbol x , \boldsymbol h)\).
Then, Proposition \ref{prp:prop2}(i) follows from
\(\REP_{(\boldsymbol X)}(\boldsymbol \theta_N) = \log[P_{\boldsymbol \theta_N} (\boldsymbol x_M) /P_{\boldsymbol \theta_N} (\boldsymbol x_m) ]\)
and the lower/upper bounds on
\(P_{\boldsymbol \theta_N} (\boldsymbol x_M)\) and
\(P_{\boldsymbol \theta_N} (\boldsymbol x_m)\) as \[
c(\boldsymbol \theta_N) \exp[\max_{\boldsymbol h}f_{\boldsymbol \theta_N} (\boldsymbol x_0 , \boldsymbol h)]
\leq P_{\boldsymbol \theta_N} (\boldsymbol x_0) \leq  P_{\boldsymbol \theta_N} (\boldsymbol x_M) \leq  2^{N_{\mathcal{H}}} c(\boldsymbol \theta_N) \exp[\max_{\boldsymbol x}\max_{\boldsymbol h}f_{\boldsymbol \theta_N} (\boldsymbol x, \boldsymbol h)]
\] and
\begin{align*}
c(\boldsymbol \theta_N) \exp[\min_{\boldsymbol x}\max_{\boldsymbol h}f_{\boldsymbol \theta_N} (\boldsymbol x , \boldsymbol h)] &\leq c(\boldsymbol \theta_N) \exp[ \max_{\boldsymbol h}f_{\boldsymbol \theta_N} (\boldsymbol x_m , \boldsymbol h)] \\
&\leq P_{\boldsymbol \theta_N} (\boldsymbol x_m) \\
&\leq P_{\boldsymbol \theta_N} (\boldsymbol x_1) \\ 
&\leq 2^{N_{\mathcal{H}}} c(\boldsymbol \theta_N) \exp[ \max_{\boldsymbol h}f_{\boldsymbol \theta_N} (\boldsymbol x_1 , \boldsymbol h)]\\&=&2^{N_{\mathcal{H}}} c(\boldsymbol \theta_N) \exp[\min_{\boldsymbol x}\max_{\boldsymbol h}f_{\boldsymbol \theta_N} (\boldsymbol x , \boldsymbol h)]
\end{align*}
To prove Proposition \ref{prp:prop2}, we next expand the function
\(f_{\boldsymbol \theta_N} (\boldsymbol x, \boldsymbol h)\) from
\eqref{eq:f} as \[
f_{\boldsymbol \theta_N} (\boldsymbol x, \boldsymbol h)= \sum_{j=1}^{N_\mathcal{H}} h_j \theta_{j,N}^{\mathcal{H}}  +   \sum_{i=1}^N \left( \theta_{i,N}^{\mathcal{V}} + \sum_{j=1}^{N_\mathcal{H}}  h_j  \theta_{ij,N}^{\mathcal{VH}}\right)x_i=\sum_{i=1}^N x_i  \theta_{i,N}^{\mathcal{V}} +  \sum_{j=1}^{N_\mathcal{H}} \left( \theta_{j,N}^{\mathcal{H}}  + \sum_{i=1}^N  x_i  \theta_{ij,N}^{\mathcal{VH}}\right)h_j.
\] By this and the fact that \(x_i,h_j\in\{\pm 1\}\), we then have
\begin{align}
\nonumber \max_{\boldsymbol x}  f_{\boldsymbol \theta_N} (\boldsymbol x, \boldsymbol h) =  \sum_{j=1}^{N_\mathcal{H}} h_j \theta_{j,N}^{\mathcal{H}}  +   a_{\boldsymbol \theta_N, \mathcal{H}} (\boldsymbol h), & \min_{\boldsymbol x}  f_{\boldsymbol \theta_N} (\boldsymbol x, \boldsymbol h) =  \sum_{j=1}^{N_\mathcal{H}} h_j \theta_{j,N}^{\mathcal{H}}  - a_{\boldsymbol \theta_N, \mathcal{H}} (\boldsymbol h),\\
\nonumber \max_{\boldsymbol h}  f_{\boldsymbol \theta_N} (\boldsymbol x, \boldsymbol h) = \sum_{i=1}^{N}x_i \theta_{i,N}^{\mathcal{V}} + b_{\boldsymbol \theta_N, \mathcal{V}} (\boldsymbol x), & \min_{\boldsymbol h}  f_{\boldsymbol \theta_N} (\boldsymbol x, \boldsymbol h) = \sum_{i=1}^{N}x_i \theta_{i,N}^{\mathcal{V}}  -   b_{\boldsymbol \theta_N, \mathcal{V}} (\boldsymbol x), \\ 
\label{eq:max}
a_{\boldsymbol \theta_N, \mathcal{H}} (\boldsymbol h) \equiv \sum_{i=1}^N \left| \theta_{i,N}^{\mathcal{V}} + \sum_{j=1}^{N_\mathcal{H}}  h_j  \theta_{ij,N}^{\mathcal{VH}}\right|, &  b_{\boldsymbol \theta_N, \mathcal{V}} (\boldsymbol x) \equiv \sum_{j=1}^{N_\mathcal{H}} \left| \theta_{j,N}^{\mathcal{H}}  + \sum_{i=1}^N  x_i  \theta_{ij,N}^{\mathcal{VH}}\right|,
\end{align}
where
\(\boldsymbol h^T \boldsymbol \theta_N^{\mathcal{H}}=\sum_{j=1}^{N_\mathcal{H}} h_j \theta_{j,N}^{\mathcal{H}}\),
\(\boldsymbol x^T \boldsymbol \theta_N^{\mathcal{V}}= \sum_{i=1}^{N}x_i \theta_{i,N}^{\mathcal{V}}\)
and
\(\Gam\equiv \max_{\boldsymbol h} a_{\boldsymbol \theta_N, \mathcal{H}} (\boldsymbol h)\).
From this, it follows that
\begin{align*}
\REP_{(\boldsymbol X, \boldsymbol H)}(\boldsymbol \theta_N) &=  \max_{\boldsymbol h}\max_{\boldsymbol x}f_{\boldsymbol \theta_N} (\boldsymbol x , \boldsymbol h) -  \min_{\boldsymbol h}\min_{\boldsymbol x}f_{\boldsymbol \theta_N} (\boldsymbol x , \boldsymbol h)\\
&= \max_{\boldsymbol h_1 }  \max_{\boldsymbol h_2 }\left[ (\boldsymbol h_1 - \boldsymbol h_2)^T \boldsymbol \theta_N^{\mathcal{H}}  +   a_{\boldsymbol \theta_N, \mathcal{H}} (\boldsymbol h_1)  + a_{\boldsymbol \theta_N, \mathcal{H}} (\boldsymbol h_2)\right],
\end{align*}
which leads to the upper bound
\(\REP_{(\boldsymbol X, \boldsymbol H)}(\boldsymbol \theta_N) \leq 2 \Gam + 2 |\boldsymbol \theta_N^{\mathcal{H}} |_1\).
Then, taking \(\boldsymbol h_1=\boldsymbol h_2\) (i.e., before
maximization) gives
\(\REP_{(\boldsymbol X, \boldsymbol H)}(\boldsymbol \theta_N) \geq 2\Gam\)
and taking \(\boldsymbol h_1=-\boldsymbol h_2\), such that
\(\boldsymbol h_1^T \boldsymbol \theta_N^{\mathcal{H}} = |\theta_N^{\mathcal{H}}|_1\),
gives
\(\REP_{(\boldsymbol X, \boldsymbol H)}(\boldsymbol \theta_N) \geq 2|\theta_N^{\mathcal{H}}|_1\);
this yields the lower bound
\(\REP_{(\boldsymbol X, \boldsymbol H)}(\boldsymbol \theta_N)\geq 2\max\{\Gam, |\boldsymbol \theta_N^{\mathcal{H}} |_1\}\).

We next consider \(\elt\) and, by \eqref{eq:max}, write
\begin{align*}
\elt &=  \max_{\boldsymbol h}\max_{\boldsymbol x}f_{\boldsymbol \theta_N} (\boldsymbol x , \boldsymbol h) -  \max_{\boldsymbol h}\min_{\boldsymbol x}f_{\boldsymbol \theta_N} (\boldsymbol x , \boldsymbol h)\\
&= \max_{\boldsymbol h_1} \min_{\boldsymbol h_2 }\left[ (\boldsymbol h_1 - \boldsymbol h_2)^T \boldsymbol \theta_N^{\mathcal{H}} + a_{\boldsymbol \theta_N, \mathcal{H}} (\boldsymbol h_1)  + a_{\boldsymbol \theta_N, \mathcal{H}} (\boldsymbol h_2)\right].
\end{align*}
Taking \(\boldsymbol h_1=\boldsymbol h_2\) and maximizing over both
\(\boldsymbol h_1,\boldsymbol h_2\) produces the upper bound
\(\elt \leq 2\Gam\). Then, using
\((\boldsymbol h_1 - \boldsymbol h_2)^T \boldsymbol \theta_N^{\mathcal{H}} + a_{\boldsymbol \theta_N, \mathcal{H}} (\boldsymbol h_2) \geq - 2|\boldsymbol \theta_N^{\mathcal{H}} |_1\)
and maximizing over \(\boldsymbol h_1\) gives
\(\elt \geq \Gam- 2|\boldsymbol \theta_N^{\mathcal{H}} |_1\), while
setting \(\boldsymbol h_1=\boldsymbol h_2^*\) for \(\boldsymbol h_2^*\)
such that
\(-(\boldsymbol h_2^*) ^T \boldsymbol \theta_N^{\mathcal{H}} + a_{\boldsymbol \theta_N, \mathcal{H}} (\boldsymbol h_2^*) = \min_{\boldsymbol h_2} [-\boldsymbol h_2^T \boldsymbol \theta_N^{\mathcal{H}} + a_{\boldsymbol \theta_N, \mathcal{H}} (\boldsymbol h_2)]\)
gives
\(\elt \geq 2 a_{\boldsymbol \theta_N, \mathcal{H}} (\boldsymbol h_2^*) \geq \Gamc \equiv \min_{\boldsymbol h} a_{\boldsymbol \theta_N, \mathcal{H}} (\boldsymbol h)\).
Finally, note that for any \(\boldsymbol h\), the triangle inequality
gives
\begin{align*}
\Gam  \equiv \max_{\boldsymbol h_1}  a_{\boldsymbol \theta_N, \mathcal{H}} (\boldsymbol h_1) &\geq
[a_{\boldsymbol \theta_N, \mathcal{H}} (\boldsymbol h)  + a_{\boldsymbol \theta_N, \mathcal{H}} (-\boldsymbol h)]/2 \\
&=
2^{-1}\sum_{i=1}^{N } \left(\left| \theta_{i,N}^{\mathcal{V}}   + \sum_{j=1}^{N_{\mathcal{H}}} h_j \theta_{ij,N}^{\mathcal{VH}} \right|
+ \left| \theta_{i,N}^{\mathcal{V}}   - \sum_{j=1}^{N_{\mathcal{H}}} h_j \theta_{ij,N}^{\mathcal{VH}} \right|\right)\\
& \geq \sum_{i=1}^{N }  \left| \theta_{i,N}^{\mathcal{V}} \right| \equiv |\boldsymbol \theta_{N}^{\mathcal{V}}|_1.
\end{align*}
\hfill \(\Box\)

\textbf{Proof of Theorem \ref{thm:something}.} Let
\(L_{\boldsymbol \theta_{N}}(\boldsymbol X) = \log[ P_{\boldsymbol \theta_{N}}(\boldsymbol X)/ \min_{\boldsymbol y \in \mathcal{X}^N} P_{\boldsymbol \theta_{N}}(\boldsymbol y) ]/\REP(\boldsymbol \theta_{N})\),
where again \(\boldsymbol X=(X_1, \dots, X_N)\) and
\(\REP(\boldsymbol \theta_{N})= \log[\max_{\boldsymbol y \in \mathcal{X}^N} P_{\boldsymbol \theta_{N}}(\boldsymbol y)/ \min_{\boldsymbol y \in \mathcal{X}^N} P_{\boldsymbol \theta_{N}}(\boldsymbol y) ]\).
As \(L_{\boldsymbol \theta_{N}}(\boldsymbol X)\in[0,1]\), convergence of
\(L_{\boldsymbol \theta_{N}}(\boldsymbol X)\) to 1 in probability under
\(P_{\boldsymbol \theta_{N}}\) is equivalent to convergence to \(1\) in
expectation under \(P_{\boldsymbol \theta_{N}}\) (i.e., convergence in
expectation implies probabilistic convergence by Markov's inequality
while probabilistic convergence implies convergence in expectation by
uniform integrability/boundedness).

For \(\epsilon \in(0,1)\), let
\(\mathcal{M}_{\epsilon, \boldsymbol \theta_N}\) denote a modal set as
in \eqref{eq:mode}. By Theorem \ref{thm:degenFOES},
\(P_{\boldsymbol \theta_N}\left( \boldsymbol X\in \mathcal{M}_{\epsilon, \boldsymbol \theta_N}\right) \rightarrow 1\)
holds as \(N \rightarrow \infty\) and, by definition of \eqref{eq:mode},
\(\boldsymbol X\in \mathcal{M}_{\epsilon, \boldsymbol \theta_N}\)
follows if and only if
\(1-L_{\boldsymbol \theta_{N}}(\boldsymbol X)<\epsilon\). Hence,
\(L_{\boldsymbol \theta_{N}}(\boldsymbol X) \stackrel{p,\mathrm{E}}{\longrightarrow} 1\)
holds under \(\boldsymbol \theta_{N}\) in Theorem \ref{thm:something}.
The convergence
\(L_{-\boldsymbol \theta_{N}}(\boldsymbol X) \stackrel{p,\mathrm{E}}{\longrightarrow} 1\)
under \(-\boldsymbol \theta_{N}\) likewise follows from Corollary
\ref{cor:sign}. \hfill \(\Box\)

\clearpage

\section*{References}\label{references}
\addcontentsline{toc}{section}{References}

\hypertarget{refs}{}
\hypertarget{ref-besag1974spatial}{}
Besag, Julian. 1974. ``Spatial Interaction and the Statistical Analysis
of Lattice Systems.'' \emph{Journal of the Royal Statistical Society.
Series B (Methodological)}. JSTOR, 192--236.

\hypertarget{ref-handcock2003assessing}{}
Handcock, Mark S. 2003. ``Assessing Degeneracy in Statistical Models of
Social Networks.'' Center for Statistics; the Social Sciences,
University of Washington. \url{http://www.csss.washington.edu/}.

\hypertarget{ref-hinton2006fast}{}
Hinton, Geoffrey E, Simon Osindero, and Yee-Whye Teh. 2006. ``A Fast
Learning Algorithm for Deep Belief Nets.'' \emph{Neural Computation} 18
(7). MIT Press: 1527--54.

\hypertarget{ref-neal1992connectionist}{}
Neal, Radford M. 1992. ``Connectionist Learning of Belief Networks.''
\emph{Artificial Intelligence} 56 (1). Elsevier: 71--113.

\hypertarget{ref-pearl985bayesian}{}
Pearl, Judea. 1985. ``Bayesian Networks: A Model of Self-Activated
Memory for Evidential Reasoning.'' UCLA Computer Science Department.

\hypertarget{ref-ruelle1999statistical}{}
Ruelle, D. 1999. \emph{Statistical Mechanics: Rigorous Results}. London:
Imperial College Press.

\hypertarget{ref-salakhutdinov2009deep}{}
Salakhutdinov, Ruslan, and Geoffrey E Hinton. 2009. ``Deep Boltzmann
Machines.'' In \emph{International Conference on Artificial Intelligence
and Statistics}, 448--55. AI \& Statistics.

\hypertarget{ref-schweinberger2011instability}{}
Schweinberger, Michael. 2011. ``Instability, Sensitivity, and Degeneracy
of Discrete Exponential Families.'' \emph{Journal of the American
Statistical Association} 106 (496). Taylor \& Francis: 1361--70.

\hypertarget{ref-smolensky1986information}{}
Smolensky, Paul. 1986. ``Information Processing in Dynamical Systems:
Foundations of Harmony Theory.'' DTIC Document.

\hypertarget{ref-wasserman1994social}{}
Wasserman, Stanley, and Katherine Faust. 1994. \emph{Social Network
Analysis: Methods and Applications}. Vol. 8. Cambridge: Cambridge
University Press.

\end{document}
